%%%%%%%%%%%%%%%%%%%%%%%%%%%%%%%%%%%%%%%%
% datoteka diploma-vzorec.tex
%
% vzorčna datoteka za pisanje diplomskega dela v formatu LaTeX
% na UL Fakulteti za računalništvo in informatiko
%
% vkup spravil Gašper Fijavž, december 2010
% 
%
%
% verzija 12. februar 2014 (besedilo teme, seznam kratic, popravki Gašper Fijavž)
% verzija 10. marec 2014 (redakcijski popravki Zoran Bosnić)
% verzija 11. marec 2014 (redakcijski popravki Gašper Fijavž)
% verzija 15. april 2014 (pdf/a 1b compliance, not really - just claiming, Damjan Cvetan, Gašper Fijavž)
% verzija 23. april 2014 (privzeto cc licenca)
% verzija 16. september 2014 (odmiki strain od roba)
% verzija 28. oktober 2014 (odstranil vpisno številko)
% verija 5. februar 2015 (Literatura v kazalu, online literatura)
% verzija 25. september 2015 (angl. naslov v izjavi o avtorstvu)
% verzija 26. februar 2016 (UL izjava o avtorstvu)
% verzija 16. april 2016 (odstranjena izjava o avtorstvu)


\documentclass[a4paper, 12pt]{book}

\usepackage[utf8x]{inputenc}   % omogoča uporabo slovenskih črk kodiranih v formatu UTF-8
\usepackage[slovene,english]{babel}    % naloži, med drugim, slovenske delilne vzorce
\usepackage[pdftex]{graphicx}  % omogoča vlaganje slik različnih formatov
\usepackage{fancyhdr}          % poskrbi, na primer, za glave strani
\usepackage{amssymb}           % dodatni simboli
\usepackage{amsmath}           % eqref, npr.
%\usepackage{hyperxmp}
\usepackage[pdftex, colorlinks=true,
						citecolor=black, filecolor=black, 
						linkcolor=black, urlcolor=black,
						pagebackref=false, 
						pdfproducer={LaTeX}, pdfcreator={LaTeX}, hidelinks]{hyperref}

\usepackage{microtype}%za lepso postavitev/delitev besed
\usepackage{cprotect}
\usepackage{listingsutf8}
\lstset{
    % osnovno
    tabsize=4,
    basicstyle={\ttfamily \scriptsize},
    identifierstyle={\ttfamily},
    inputencoding=cp1250,
    %
    % barvanje kode
    commentstyle={\sffamily \color[rgb]{0,0.5,0}},
    stringstyle=\color[rgb]{0.5,0,1},
    keywordstyle=\color[rgb]{0,0,1},
    language=Python,
    showstringspaces=false,
    showspaces=false,
    %
    % iz source datotek potegnemo tisto med //@begin@ ...koda... //@end@
    % rangeprefix={\#@},
    % rangesuffix={@},
    % linerange=begin-end,
    % includerangemarker=false,
    %
    % lomljenje predolgih vrstic
    breaklines=true,
    prebreak=\raisebox{0ex}[0ex][0ex]{\ensuremath{\hookleftarrow}},
    %
    % okvir
    frame=single,
    frameround={t}{t}{t}{t},
    xleftmargin=23pt,
    xrightmargin=4pt,
    framexleftmargin=19pt,
    framerule=0pt,
    %
    % številčenje
    numbers=left,
    firstnumber=1,
    extendedchars=true,
}

%%%%%%%%%%%%%%%%%%%%%%%%%%%%%%%%%%%%%%%%
%	DIPLOMA INFO
%%%%%%%%%%%%%%%%%%%%%%%%%%%%%%%%%%%%%%%%
\newcommand{\ttitle}{Vzorec diplomskega dela}
\newcommand{\ttitleEn}{Diploma thesis sample}
\newcommand{\tsubject}{\ttitle}
\newcommand{\tsubjectEn}{\ttitleEn}
\newcommand{\tauthor}{Matjaž Kralj}
\newcommand{\tkeywords}{računalnik, računalnik, računalnik}
\newcommand{\tkeywordsEn}{computer, computer, computer}



\usepackage{hyperref}
%%%%%%%%%%%%%%%%%%%%%%%%%%%%%%%%%%%%%%%%
%	HYPERREF SETUP
%%%%%%%%%%%%%%%%%%%%%%%%%%%%%%%%%%%%%%%%
\hypersetup{pdftitle={\ttitle}}
\hypersetup{pdfsubject=\ttitleEn}
\hypersetup{pdfauthor={\tauthor, matjaz.kralj@fri.uni-lj.si}}
\hypersetup{pdfkeywords=\tkeywordsEn}

\usepackage{emptypage}
\usepackage{color}
\usepackage{float}
\newfloat{snippet}{thp}{lop}
\floatname{snippet}{Primer}

\newcommand\angl[1]{({angl.\ }{\it#1})}

\newcommand\fnurl[1]{%
    \footnote{\url{#1}}%
}

%%%%%%%%%%%%%%%%%%%%%%%%%%%%%%%%%%%%%%%%
% postavitev strani
%%%%%%%%%%%%%%%%%%%%%%%%%%%%%%%%%%%%%%%%  

\addtolength{\marginparwidth}{-20pt} % robovi za tisk
\addtolength{\oddsidemargin}{40pt}
\addtolength{\evensidemargin}{-40pt}

\renewcommand{\baselinestretch}{1.3} % ustrezen razmik med vrsticami
\setlength{\headheight}{15pt}        % potreben prostor na vrhu
\renewcommand{\chaptermark}[1]%
{\markboth{\MakeUppercase{\thechapter.\ #1}}{}} \renewcommand{\sectionmark}[1]%
{\markright{\MakeUppercase{\thesection.\ #1}}} \renewcommand{\headrulewidth}{0.5pt} \renewcommand{\footrulewidth}{0pt}
\fancyhf{}
\fancyhead[LE,RO]{\sl \thepage} \fancyhead[LO]{\sl \rightmark} \fancyhead[RE]{\sl \leftmark}



\newcommand{\BibTeX}{{\sc Bib}\TeX}

%%%%%%%%%%%%%%%%%%%%%%%%%%%%%%%%%%%%%%%%
% naslovi
%%%%%%%%%%%%%%%%%%%%%%%%%%%%%%%%%%%%%%%%  


\newcommand{\autfont}{\Large}
\newcommand{\titfont}{\LARGE\bf}
\newcommand{\clearemptydoublepage}{\newpage{\pagestyle{empty}\cleardoublepage}}
\setcounter{tocdepth}{1}	      % globina kazala

%%%%%%%%%%%%%%%%%%%%%%%%%%%%%%%%%%%%%%%%
% konstrukti
%%%%%%%%%%%%%%%%%%%%%%%%%%%%%%%%%%%%%%%%  
\newtheorem{izrek}{Izrek}[chapter]
\newtheorem{trditev}{Trditev}[izrek]
\newenvironment{dokaz}{\emph{Dokaz.}\ }{\hspace{\fill}{$\Box$}}

%%%%%%%%%%%%%%%%%%%%%%%%%%%%%%%%%%%%%%%%%%%%%%%%%%%%%%%%%%%%%%%%%%%%%%%%%%%%%%%
%% PDF-A
%%%%%%%%%%%%%%%%%%%%%%%%%%%%%%%%%%%%%%%%%%%%%%%%%%%%%%%%%%%%%%%%%%%%%%%%%%%%%%%

%%%%%%%%%%%%%%%%%%%%%%%%%%%%%%%%%%%%%%%% 
% define medatata
%%%%%%%%%%%%%%%%%%%%%%%%%%%%%%%%%%%%%%%% 
\def\Title{\ttitle}
\def\Author{\tauthor, matjaz.kralj@fri.uni-lj.si}
\def\Subject{\ttitleEn}
\def\Keywords{\tkeywordsEn}

%%%%%%%%%%%%%%%%%%%%%%%%%%%%%%%%%%%%%%%% 
% \convertDate converts D:20080419103507+02'00' to 2008-04-19T10:35:07+02:00
%%%%%%%%%%%%%%%%%%%%%%%%%%%%%%%%%%%%%%%% 
\def\convertDate{%
    \getYear
}

{\catcode`\D=12
 \gdef\getYear D:#1#2#3#4{\edef\xYear{#1#2#3#4}\getMonth}
}
\def\getMonth#1#2{\edef\xMonth{#1#2}\getDay}
\def\getDay#1#2{\edef\xDay{#1#2}\getHour}
\def\getHour#1#2{\edef\xHour{#1#2}\getMin}
\def\getMin#1#2{\edef\xMin{#1#2}\getSec}
\def\getSec#1#2{\edef\xSec{#1#2}\getTZh}
\def\getTZh +#1#2{\edef\xTZh{#1#2}\getTZm}
\def\getTZm '#1#2'{%
    \edef\xTZm{#1#2}%
    \edef\convDate{\xYear-\xMonth-\xDay T\xHour:\xMin:\xSec+\xTZh:\xTZm}%
}

\expandafter\convertDate\pdfcreationdate 

%%%%%%%%%%%%%%%%%%%%%%%%%%%%%%%%%%%%%%%%
% get pdftex version string
%%%%%%%%%%%%%%%%%%%%%%%%%%%%%%%%%%%%%%%% 
\newcount\countA
\countA=\pdftexversion
\advance \countA by -100
\def\pdftexVersionStr{pdfTeX-1.\the\countA.\pdftexrevision}


%%%%%%%%%%%%%%%%%%%%%%%%%%%%%%%%%%%%%%%%
% XMP data
%%%%%%%%%%%%%%%%%%%%%%%%%%%%%%%%%%%%%%%%  
%\usepackage{xmpincl}
%\includexmp{pdfa-1b}

%%%%%%%%%%%%%%%%%%%%%%%%%%%%%%%%%%%%%%%%
% pdfInfo
%%%%%%%%%%%%%%%%%%%%%%%%%%%%%%%%%%%%%%%%  
\pdfinfo{%
    /Title    (\ttitle)
    /Author   (\tauthor)
    /Subject  (\ttitleEn)
    /Keywords (\tkeywordsEn)
    /ModDate  (\pdfcreationdate)
    /Trapped  /False
}


%%%%%%%%%%%%%%%%%%%%%%%%%%%%%%%%%%%%%%%%%%%%%%%%%%%%%%%%%%%%%%%%%%%%%%%%%%%%%%%
%%%%%%%%%%%%%%%%%%%%%%%%%%%%%%%%%%%%%%%%%%%%%%%%%%%%%%%%%%%%%%%%%%%%%%%%%%%%%%%

\begin{document}
\selectlanguage{slovene}
\frontmatter
\setcounter{page}{1} %
\renewcommand{\thepage}{}       % preprecimo težave s številkami strani v kazalu

%%%%%%%%%%%%%%%%%%%%%%%%%%%%%%%%%%%%%%%%
%naslovnica
 \thispagestyle{empty}%
   \begin{center}
    {\large\sc Univerza v Ljubljani\\%
      Fakulteta za računalništvo in informatiko}%
    \vskip 10em%
    {\autfont Miha Zidar \par}%
    {\titfont Dostop do podatkov Svetovne banke v orodju Orange \par}%
    {\vskip 2em \textsc{DIPLOMSKO DELO\\[2mm] 
    UNIVERZITETNI ŠTUDIJSKI PROGRAM RAČUNALNIŠTVO IN INFORMATIKA}\par}%
    \vfill\null%
    {\large \textsc{Mentor}: prof. dr. Blaž Zupan \par}%
    {\vskip 2em \large Ljubljana, 2016 \par}%
\end{center}
% prazna stran
\clearemptydoublepage

%%%%%%%%%%%%%%%%%%%%%%%%%%%%%%%%%%%%%%%%
%copyright stran
\thispagestyle{empty}
\vspace*{8cm}
Fakulteta za računalništvo in informatiko podpira javno dostopnost znanstvenih, strokovnih in razvojnih rezultatov. Zato priporoča objavo dela pod katero od licenc, ki omogočajo prosto razširjanje diplomskega dela in/ali možnost nadaljne proste uporabe dela. Ena izmed možnosti je izdaja diplomskega dela pod katero od Creative Commons licenc \href{http://creativecommons.si}{http://creativecommons.si}

Morebitno pripadajočo programsko kodo praviloma objavite pod, denimo, licenco 
\emph{GNU General Public License, različica 3}. Podrobnosti licence so dostopne na spletni strani \href{http://www.gnu.org/licenses/}{http://www.gnu.org/licenses/}.

\begin{center}
\mbox{}\vfill
\emph{Besedilo je oblikovano z urejevalnikom besedil \LaTeX.}
\end{center}
% prazna stran
\clearemptydoublepage

%%%%%%%%%%%%%%%%%%%%%%%%%%%%%%%%%%%%%%%%
% stran 3 med uvodnimi listi
\thispagestyle{empty}
\vspace*{4cm}

\noindent
Fakulteta za računalništvo in informatiko izdaja naslednjo nalogo:
\medskip
\begin{tabbing}
\hspace{32mm}\= \hspace{6cm} \= \kill




Tematika naloge:
\end{tabbing}
Na spletu je veliko odprtih baz podatkov, za katere bi bilo koristno, da bi bile dostopne v orodjih za podatkovno analitiko. Primer take baze so spletne strani in programski vmesniki Svetovne banke, preko katerih lahko dostopamo do demografskih, gospodarskih in klimatskih podatkov. V nalogi razvijte knjižnico in komponente z grafičnimi vmesnikom za program Orange, s katerimi lahko enostavno in hitro pridobimo podatke iz Svetovne banke in jih z obstoječimi analitičnimi gradniki v Orange-u tudi analiziramo.
\vspace{15mm}






\vspace{2cm}

% prazna stran
\clearemptydoublepage

% zahvala
\thispagestyle{empty}\mbox{}\vfill\null\it%
Zahvalil bi se mentorju, prof.\ dr.\ Blažu Zupanu in članom laboratorija za 
bioinformatiko za pomoč in usmerjanje med izdelavo diplomskega dela. 
Prav tako bi se zahvalil svojemu partnerju, staršem in prijateljem za
spodbudo.
\rm\normalfont

% prazna stran
\clearemptydoublepage

%%%%%%%%%%%%%%%%%%%%%%%%%%%%%%%%%%%%%%%%%
%% posvetilo
%\thispagestyle{empty}\mbox{}{\vskip0.20\textheight}\mbox{}\hfill\begin{minipage}{0.55\textwidth}%
%% Za bubija
%\normalfont\end{minipage}
%
%% prazna stran
%\clearemptydoublepage

%%%%%%%%%%%%%%%%%%%%%%%%%%%%%%%%%%%%%%%%
% kazalo
\pagestyle{empty}
\def\thepage{}% preprecimo tezave s stevilkami strani v kazalu
\tableofcontents{}


% prazna stran
\clearemptydoublepage

%%%%%%%%%%%%%%%%%%%%%%%%%%%%%%%%%%%%%%%%
% seznam kratic

% \chapter*{Seznam uporabljenih kratic}
% 
% \begin{tabular}{l|l|l}
%   {\bf kratica} & {\bf angleško} & {\bf slovensko} \\ \hline
%   % after \\: \hline or \cline{col1-col2} \cline{col3-col4} ...
%   {\bf CA} & classification accuracy & klasifikacijska točnost \\
%   {\bf DBMS} & database management system & sistem za upravljanje podatkovnih baz \\
%   {\bf SVM} & support vector machine & metoda podpornih vektorjev \\
%   \dots & \dots & \dots \\
% \end{tabular}



% prazna stran
\clearemptydoublepage

%%%%%%%%%%%%%%%%%%%%%%%%%%%%%%%%%%%%%%%%
% povzetek
%%%%%%%%%%%%%%%%%%%%%%%%%%%%%%%%%%%%%%%%
% povzetek 
\addcontentsline{toc}{chapter}{Povzetek}
\chapter*{Povzetek}



\textbf{Naslov:} Dostop do podatkov Svetovne banke v orodju Orange

\ \\
\textbf{Avtor:} Miha Zidar

\ \\
Program Orange je prosto dostopno orodje za podatkovno rudarjenje, v katerem
lahko uporabimo razli"cne podatkovne vire. Obstoje"ci dodatki (vkljucki?) "ze
omogo"cajo pridobivanje 
omogo"cajo izbrati predpripravljenih testnih podatkov, 

\ \\
\textbf{Ključne besede:} Programski vmesnik, gospodarski indikatorji, podnebje,
Orange



uoet


% prazna stran
\clearemptydoublepage

%%%%%%%%%%%%%%%%%%%%%%%%%%%%%%%%%%%%%%%%
% abstract
\selectlanguage{english}
\addcontentsline{toc}{chapter}{Abstract}
\chapter*{Abstract}



\subsubsection*{Keywords:}

rainbow

\selectlanguage{slovene}


\selectlanguage{slovene}
% prazna stran
\clearemptydoublepage

%%%%%%%%%%%%%%%%%%%%%%%%%%%%%%%%%%%%%%%%
\mainmatter
\setcounter{page}{1}
\pagestyle{fancy}


\chapter{Uvod}

Zivimo v "casu ko je na spletu zmeraj vec prosto dostopnih programski vmesnikov
za razne podatkovne baze. Pri zmeraj ve"cjem "stevilu podatkov pa nam primanjkuje
Orodij za obdelavo le teh podatkov. 
Danes imamo zmeraj ve"c podatkov

\section{Motivacija}

BBBBBB

\section{Cilji}

CCCCC


\chapter{Podatkovne zbirke Svetovne Banke}

Pri diplomski nalogi smo se osredotočili na dva programska vmesnika za dostop 
podatkov Svetovne banke. To sta ``ClimateAPI'', s katerim dostopamo do 
podatkovne zbirke meteoroloških meritev in ``IndicatorAPI'', s katerim dostopamo do 
zbirke podatkov raznih indikatorjev stopenj razvoja držav.
Za uporabo podatkovne zbirke Svetovne banke smo se odločili, ker združuje in na
enovit način predstavi podatke iz več različnih virov. Podatkovni viri za 
indikatorje stopnje razvoja držav so:
\begin{itemize}  
  \item Svetovni indikatorji razvoja~\cite{world_dev_ind}, % World Development Indicators, 
  \item Globalni finančni razvoj~\cite{glob_fin_dev},
  \item Afriški indikatorji razvoja~\cite{africa_dev_ind},
  \item Poslovanje~\cite{doing_buseness},
  \item Podjetniške raziskave~\cite{ent_surveys}, 
  \item Razvojni cilji~\cite{mil_dev_goals}, 
  \item Statistike izobraževanja~\cite{edu_stat}, 
  \item Statistike spolov~\cite{gen_stat},
  \item Statistike zdravja in prehranjevanja~\cite{health_pop_stat} in
  \item Rezultati meritev IDA~\cite{ida_res_mes_sys}.
\end{itemize}  

Podatkovni vir zbirke podnebnih meritev pa je osnovan na podatkih oddelka
za podnebne raziskave \angl{Climatic Research Unit}~\cite{climate_data}.

Svetovna banka omogoča dostop do podatkov preko programskega vmesnika 
predstavitvene arhitekture za prenos podatkov REST 
\angl{Representational State Transfer}, ki
ponuja veliko možnosti za iskanje in izbor rezultatov programskih poizvedb. Pri vsaki 
poizvedbi REST lahko določimo želeno obliko odgovora. Za poizvedbe o 
informacijah indikatorjev sta na voljo obliki 
razširljivega označevalnenga jezika XML \angl{Extensible Markup Language} 
in javascript objektne notacije JSON \angl{JavaScript Object Notation}. Programski vmesnik 
meteoroloških meritev pa ponuja samo obliko JSON. Za konsistentnost in lažjo
berljivost smo na obeh programskih vmesnikih uporabili obliko JSON. To na
programskem vmesniku indikatorjev dosežemo tako da nastavimo parameter GET
\verb|format| na vrednost \verb|json|. 


\section{Podatki indikatorjev razvoja držav}
\label{sec:podatki_ind_razvoja}



Programski vmesnik indikatorjev razvoja držav Svetovne banke omogoča dostop
do podatkov preko 16.000 raznih indikatorjev. Podatki indikatorjev so merjeni v
mesečnem, četrtletnem ali letnem intervalu. Začetek meritev podatkov
posameznega indikatorja je odvisna od vira podatkov. Najstarejši podatki segajo
do leta 1960. Poleg podatkov indikatorjev nam ta programski vmesnik omogoča 
tudi dostop do večine metapodatkov, s katerimi lahko presejamo in natančneje
določimo našo poizvedbo in ki vklju"cujejo::
\begin{itemize}
\item viri podatkov in njihovi opisi 
  \angl{Catalog Source Queries~\fnurl{http://api.worldbank.org/sources?format=json}},
\item seznam držav, skupin držav in regij z identifikatorji 
  \angl{Country Queries~\fnurl{http://api.worldbank.org/countries?format=json}},
\item razdelitev višin dohodkov z identifikatorji 
  \angl{Income Level Queries~\fnurl{http://api.worldbank.org/incomeLevels?format=json}},
\item seznam indikatorjev 
  \angl{Indicator Queries~\fnurl{http://api.worldbank.org/indicators?format=json}},
\item seznam tipov posojil 
  \angl{Lending Type Queries~\fnurl{http://api.worldbank.org/lendingTypes?format=json}},
\item seznam tem 
  \angl{Topics~\fnurl{http://api.worldbank.org/topics}}.
\end{itemize}

Za pridobitev podatkov indikatorjev potrebujemo metapodatke o indikatorjih in
državah. Primere teh metapodatkov si bomo podrobneje pogledali v nadaljevanju.

Ker je mogoče z eno poizvedbo dostopati do velike količine podatkov, ima
programski vmesnik za dostop do podatkov indikatorjev implementirano
% "stevil"cenje,
% oštevilčenje,
paginacijo,
% deljenje na strani,
s katero je omejeno število podatkov, ki jih lahko dobimo z eno
poizvedbo. Tako so podatki razdeljeni na skupine, ki jih imenujemo strani.

Vsi odgovori na veljavne poizvedbe po podatkih in metapodatkih, ki so na voljo
s programskim vmesnikom indikatorjev razvoja, imajo enako osnovno obliko. 
Poizvedbe vračajo seznam z dvema elementoma, kjer ima prvi element 
informacije o količini podatkov in trenutnem izboru podatkov, drugi element 
pa vsebuje seznam izbranih podatkov (primer \ref{basic_response}). Privzeta
vrednost števila elementov na stran je $50$, kar lahko spremenimo tako, da
poizvedbi nastavimo parameter GET \verb|per_page| na poljubno vrednost. Če
želimo pridobiti podatke z več strani, moramo za vsako stran poslati novo
poizvedbo, v kateri podamo številko želene strani s parametrom GET \verb|page|.
Veljavne poizvedbe s sitom, ki ne vrača nobenih podatkov, imajo vrednost 
drugega elementa osnovnega seznama \verb|null|.
Za neveljavne poizvedbe pa programski vmesnik vrača seznam z enim elementom,
ki vsebuje podatke o napaki poizvedbe (primer \ref{error_response}).


\begin{snippet}
\begin{center}
\begin{lstlisting}
[
    {
        'page': 1,
        'pages': 137,
        'per_page': '50',
        'total': 6831
    },
    [
        <podatki>,
        ...
    ]
]
\end{lstlisting}
\end{center}
\caption{Osnovna oblika odgovora programskega vmesnika Svetovne banke za
veljavno poizvedbo indikatorjev.} 
\label{basic_response}
\end{snippet} 


\begin{snippet}
\begin{center}
\begin{lstlisting}
[
    {
        'message': [
            {
                'id': '120',
                'key': 'Parameter \'country\' has an invalid value',
                'value': 'The provided parameter value is not valid'
            }
        ]
    }
]
\end{lstlisting}
\end{center}
\caption{Osnovna oblika odgovora programskega vmesnika Svetovne banke za
neveljavne poizvedbe.}
\label{error_response}
\end{snippet} 


\subsection{Opis seznama indikatorjev}

Programski vmesnik Svetovne banke za indikatorje razvoja nam ponuja seznam 
vseh indikatorjev z imeni, opisi, kodami in drugimi metapodatki 
(primer \ref{indicator_response}). Programski vmesnik nam omogoča tudi dostop
do podatkov posameznega indikatorja določenega s kodo in presejanje seznama 
indikatorjev glede na vir podatkov \ref{indicator_queries}. V našem programu
smo uporabili le poizvedbo za celoten seznam indikatorjev, da smo omogočili 
iskanje in presejanje po vseh poljih indikatorjev.

\begin{snippet}
\begin{center}
\begin{lstlisting}
http://api.worldbank.org/indicators?format=json
http://api.worldbank.org/indicators?format=json&source=5
http://api.worldbank.org/indicators/A10i?format=json
\end{lstlisting}
\end{center}
\caption{Primeri poizvedb po seznamu indikatorjev.
1) seznam vseh indikatorjev, 2) seznam indikatorjev glede na vir podatkov,
3) podatki indikatorja ``A10i''}
\label{indicator_queries}
\end{snippet} 


\begin{snippet}
\begin{center}
\begin{lstlisting}
{
    'id': '1.0.HCount.2.5usd',
    'name': 'Poverty Headcount (\$2.50 a day)',
    'source': {
        'id': '37',
        'value': 'LAC Equity Lab'
    },
    'sourceNote': 'The poverty headcount index measures the 
                   proportion of the population with daily per 
                   capita income (in  2005 PPP) below the poverty
                   line.',
    'sourceOrganization': 'LAC Equity Lab tabulations of SEDLAC 
                           (CEDLAS and the World Bank).',
    'topics': [
        {
            'id': '11',
            'value': 'Poverty '
        }
    ]
}
\end{lstlisting}
\end{center}
\caption{Podatki indikatorja 
% TODO
% Ali moram ime indikatorja oznaciti (naprimer velika zacetnica)
% > Podatki indikatorja Stopnja rev...
stopnja revščine pri dohodku 2,5 dolarja na dan.}
\label{indicator_response}
\end{snippet} 

\subsection{Opis seznama držav}

Seznam držav na programskem vmesniku Svetovne banke vsebuje podatke o imenih, 
opisih, ISO-3166-1 alpha kodah, regijah in druge metapodatke 
(primer \ref{country_response}). Programski
vmesnik nam omogoča tudi presejanje seznama držav po kodi države, regiji,
višini dohodka in tipu posojil (primer \ref{country_queries})

% \begin{description}
% \item [id] koda države, regije ali skupine držav,
% \item [region] regija,
% \item [incomeLevel] višina dohodka,
% \item [lendingType] tipov posojil. % TODO preveri prevod!
% \end{description}

\begin{snippet}
\begin{center}
\begin{lstlisting}
http://api.worldbank.org/countries?format=json
http://api.worldbank.org/countries/svn?format=json
http://api.worldbank.org/countries?format=json&incomeLevel=HIC&region=ECS
\end{lstlisting}
\end{center}
\caption{Primeri poizvedb po seznamu držav.
1) seznam vseh držav, 2) podatki ene države,
3) seznam držav v Evropi in Osrednji Aziji z visoko višino dohodka.}
\label{country_queries}
\end{snippet} 

Ta seznam ne vsebuje zgolj samo držav, ampak tudi regije in skupine držav, 
združenih glede na različne kriterije (višine dohodka, velikost, stopnja
razvoja). Poleg tega zgornji seznam vsebuje tudi nekatere izjeme kot je trenutno
Kosovo. V nadaljevanju bomo za vse naštete tipe lokacijskih podatkov
uporabljali besedo ``države''. 

\begin{snippet}
\begin{center}
\begin{lstlisting}
{
    'id': 'ABW',
    'iso2Code': 'AW',
    'name': 'Aruba',
    'region': {
        'id': 'LCN',
        'value': 'Latin America & Caribbean '
    },
    'adminregion': {
        'id': '',
        'value': ''
    },
    'incomeLevel': {
        'id': 'HIC',
        'value': 'High income'
    },
    'lendingType': {
        'id': 'LNX',
        'value': 'Not classified'
    },
    'capitalCity': 'Oranjestad',
    'longitude': '-70.0167',
    'latitude': '12.5167'
},
\end{lstlisting}
\end{center}
\caption[some]{Izsek podatkov veljavne poizvedbe držav.}
\label{country_response}
\end{snippet} 


\subsection{Dostop do podatkov indikatorjev}

Za dostop do podatkov posameznega indikatorja potrebujemo kodo
indikatorja s seznama vseh indikatorjev in kodo ene ali več držav. Namesto
kode ene ali več držav, lahko uporabimo tudi ključno besedo ``all'', ki
označuje vse kode držav. Pri večjih količinah podatkov lahko z dodatnimi
parametri določimo število podatkov na stran, in želeno stran podatkov.
Primer \ref{indicator_dataset_request} prikazuje osnovno obliko poizvedbe,
kjer so:
\begin{description}
\item [country] s podpičjem ločen seznam kod izbranih držav, ki jih 
	  preberemo iz polja ``id'' ali ``iso2Code'', ki sta prikazana v Primeru 
    \ref{country_response}, ali pa ključna beseda ``all'',
\item [indicator\_id] polje ``id'' indikatorja ki je prikazano v Primeru 
    \ref{indicator_response},
\item [parametri] Dodatni parametri GET 
\end{description}
Za poizvedbe do podatkov indikatorjev so poleg osnovnih parametrov GET 
\verb|per_page|, \verb|page| in \verb|format|, opisanih v poglavju 
\ref{sec:podatki_ind_razvoja}, na voljo tudi dodatni parametri za presejanje
rezultatov poizvedbe:
\begin{description}  
\item [MRV] Številska vrednost, ki določi maksimalno število zadnjih meritev,
    ki jih programski vmesnik vrne. Ko uporabljamo polje \verb|mrv| bo 
    programski vmesnik izpustil ničelne vrednosti za obdobja v katerih ni
    meritev.
\item [gapfill] Zastavica \verb|'y'| ali \verb|'n'| za manjkajoče vrednosti meritev.
    Vrednost \verb|'y'| kombinaciji s poljem \verb|mrv| poskrbi da programski 
    vmesnik ne izpusti nobenega časovnega intervala.
\item [date] Polje oblike \verb|'leto'| ali \verb|'leto:leto'| ki omeji rezultate poizvedbe
    na določeno leto ali interval med določenimi leti. 
\end{description}


\begin{snippet}
\begin{center}
\begin{lstlisting}
http://api.worldbank.org/en/countries/<country>/indicators/<indicator_id>?<parametri>
\end{lstlisting}
\end{center}
\caption{Osnovna oblika poizvedbe za podatke enega indikatorja.}
\label{indicator_dataset_request}
\end{snippet} 

% http://api.worldbank.org/countries/svn/indicators/SL.TLF.ACTI.FE.ZS?format=json
% http://api.worldbank.org/countries/svn;usa/indicators/SL.TLF.ACTI.FE.ZS?format=json
% http://api.worldbank.org/countries/all/indicators/SL.TLF.ACTI.FE.ZS?format=json

Privzeta vrednost za količino podatkov na stran \verb|per_page| je $50$. 
Zgornja meja pa ni strogo določena, vendar je odvisna od velikosti odgovora. 
Ugotovili smo, da se zanesljivost programskega vmesnika manjša z večjo 
količino podatkov na stran. V našem programu smo se omejili na $1000$ podatkov
na stran, kar se je izkazalo za uporabno razmerje med hitrostjo in 
zanesljivostjo programskega vmesnika. Privzeto bo programski vmesnik vrnil 
podatke za vse časovne vrednosti. V odgovoru API-ja dobimo seznam objektov 
(primer \ref{dataset_response}) z datumom, indikatorjem, državo in vrednostjo.

\begin{snippet}
\begin{center}
\begin{lstlisting}
{
    'indicator': {
        'id': 'SP.POP.TOTL',
        'value': 'Population, total'
    },
    'country': {
        'id': 'IL',
        'value': 'Israel'
    },
    'value': '6289000',
    'decimal': '0',
    'date': '2000'
}
\end{lstlisting}
\end{center}
\caption{Podatki za indikator SP.POP.TOTL (skupno število prebivalcev države) za Izrael leta
2000.}
\label{dataset_response}
\end{snippet} 

Slabosti programskega vmesnika indikatorjev Svetovne banke za uporabo v namene
podatkovnega rudarjenja so v tem, da vmesnik ni namenjen prenosu večje 
količine podatkov z eno samo poizvedbo. Zaradi ostranjevanja moramo za en sam 
indikator narediti več poizvedb, da prenesemo podatke z vseh strani. Prav 
tako podatkovni vmesnik ne podpira poizvedb po več indikatorjih hkrati, kar
potrebujemo za iskanje zakonitosti med posameznimi indikatorji.

\section{Podatki podnebnih meritev}

Programski vmesnik Svetovne banke za podnebne podatke omogoča dostop do 
podatkov napovednih modelov in zgodovinskih meritev meteoroloških postaj. V tej 
diplomski nalogi smo se odločili uporabiti samo podatke zgodovinskih meritev, 
saj si s temi podatki lahko uporabnik programa Orange sam sestavi svoje 
napovedne modele.

Za razliko od uporabe programskega vmesnika indikatorjev, lahko pri tem
programskem vmesniku uporabljamo veljavne ISO 3166-1 alpha-2 ali ISO 3166-1 
alpha-3 kode držav, ali pa številski identifikator vodotočnega 
območja.

Ta programski vmesnik nam omogoča dostop do podatkov o povprečnih 
temperaturah in padavinah v časovnih obdobjih enega leta, desetletja ali pa 
nam omogoča dostop do mesečnih povprečij skozi vsa leta meritev.


\subsection{Dostop do podatkov podnebnih meritev}

Za dostop do podnebnih podatkov preko programskega vmesnika Svetovne banke
potrebujemo ISO-3166-1 alpha-3 kodo države ali številski identifikator
vodotočnega območja (Slika \ref{climate_data_api_basins}). Programski vmesnik
nam omogoča dostop do meritev povprečnih količin padavin in temperatur za 
letno ali desetletno obdobje. Poleg letnega in desetletnega obdobja pa nam 
programski vmesnik ponuja tudi povprečno količino padavin in temperatur za 
posamezne mesece skozi vsa leta meritev. Obliko poizvedbe prikazuje primer 
\ref{climate_dataset_request}, kjer je:
\begin{description}
\item [loc\_type] vrsta identifikatorja območja (``basin'' za vodotočno območje, 
  ``country'' za države),
\item [data\_type] vrsta meritev (``pr'' za padavine, ``tas'' za temperature),
\item [interval] vrsta meritvenega obdobja (``month'' za mesečno, ``year'' za letno in
  ``decade'' za desetletno),
\item [location] koda države ali številski identifikator vodotočnega območja.
\end{description}
Za razliko od programskega vmesnika indikatorjev, nam programski vmesnik
podnebnih meritev z eno poizvedbo omogoča dostop do podatkov le za eno državo.
To pomeni, da je količina podatkov dovolj omejena, da nam programski vmesnik
vedno vrne vse podatke brez ostranjevanja, kot prikazuje primer 
\ref{climate_dataset_response},


\begin{figure}
\begin{center}
\includegraphics[width=13.75cm]{pic/climate_data_api_basins.pdf}
\end{center}
\caption{Prikaz vodotočnih območij sveta.}
\label{climate_data_api_basins}
\end{figure} 


\begin{snippet}
\begin{center}
\begin{lstlisting}
http://climatedataapi.worldbank.org/climateweb/rest/v1/<loc_type>/cru/<data_type>/<interval>/<location>
\end{lstlisting}
\end{center}
\caption{Osnovna oblika poizvedbe za podnebne podatke.}
\label{climate_dataset_request}
\end{snippet} 


\begin{snippet}
\begin{center}
\begin{lstlisting}
[
    {
        'month': 0,
        'data': 68.93643
    },
    {
        'month': 1,
        'data': 64.23069
    },
    {
        'month': 2,
        'data': 81.098724
    },
    ...
]
\end{lstlisting}
\end{center}
\caption{Primer odgovora za poizvedbo količine padavin v posameznih mesecih v 
  Sloveniji.}
\label{climate_dataset_response}
\end{snippet} 






\section{Težave pri uporabi programskih vmesnikov Svetovne banke}
\label{api_gotchas}


Programski vmesniki Svetovne banke zajemajo podatke iz različnih virov, zato je
težko zagotoviti pravilnost in konsistentnost podatkov. Poleg tega pa se 
programski vmesnik in spletna stran z dokumentacijo občasno spremenita, kar
povzroča še dodatne težave pri uporabi. Nekatere težave, ki smo jih opazili
so:
\begin{itemize}  
\item nekaterim delom dokumentacije se je med izdelavo te diplomske naloge
  spremenil spletni naslov, tako da do tistih delov sedaj nimamo več dostopa,
\item polje za datum \verb|date| v odgovoru je opisano, vendar niso dokumentirane vse možne vrednosti
    (nekaj primerov nedokumentiranih vrednosti:
    ``last known value'' ``2001 - 2015'' ``2040''),
\item delovanje sita z različnimi kombinacijami polj \verb|mrv|, \verb|gapfill|
  in \verb|date| ni ustrezno opisano,
\item v odgovoru poizvedbe po podatkih indikatorjev ponekod manjkajo vrednosti
  kot so koda države, ime države ali ime indikatorja,
\item zgornja meja števila izbranih lokacij na $250$ ni navedena, prav tako pa ni
  dokumentirana napaka, ki jo v tem primeru vrne programski vmesnik
\item nemogoče je ugotoviti pogostost vzorčenja indikatorja \verb|frequency|, .
\end{itemize}  









\chapter{Knjiznica in gradniki za Orange}

V okviru diplomske naloge smo razvili tri locene komponente za programerje in
koncne uporabnike programa Orange. 

Prva komponenta je programska knjiznica simple\_wbd, ki
omogoca enostaven dostop do programskega vmesnika indikatorjev in klimatskih
podatkov Svetovne banke. Ta knjiznica je narejena s cim manj odvisnosti in je 
namenjena splosni uporabi v python programih. Poudarka pri zasnovi knjiznice 
simple\_wbd sta predvsem enostavnost razsiritve in zanesljivost. Ta cilja
dosezemo z mehanizmom za vkljucevanje lastne kode v komponente knjiznice
in mehanizmi za popravljanje ali odstranjevanje pokvarjenih podatkov.

Drugi sestavni del je razsiritev knjiznice simple\_wbd s funkcionalnostmi, 
potrebnimi za lazje delo v programu Orange. To predvsem zavzema pretvorbo
pridobljenih podatkov v Orange in numpy tabele. Ta sklop je namenjen skriptnemu
delu s programom Orange in je dostopen kot api\_wrapper python modul. 

Tretji sestavni del je graficni vmesnik za uporabo api\_wrapper modula. Namen
graficnega vmesnika je omogociti ne-programerjem dostop do podatkov 
programskega vmesnika Svetovne banke znotraj programa Orange za namen obdelave,
analize in iskanja zakonitosti med podatki.

\section{Knjiznica simple\_wbd}

Knjiznica simple\_wbd programerjem olajsa dostop do podatkov programskega 
vmesnika Svetovne banke. Glavna lastnost te knjiznice je zdruzevanje vecjega 
stevila zahtev po podatkih in enostavna predstavitev dobljenih podatkov. 
Druga lastnost je pretvorba podatkov iz vec dimenzij v dvo-dimenzionalno polje,
primerno za uporabo v programu Orange. Glavna vmesnika te knjiznice sta 
IndicatorAPI in ClimateAPI. Prvi omogoca pridobivanje podatkov iz programskega 
vmesnika indikatorjev Svetovne banke, drugi pa s programskega vmesnika
podnebnih meritev.


%% % moznost razsiritve z dedovanjem dataset razreda.
%% % 
%% 
%% - omogoca pridobivanje vrednosti za filtre:
%%     - indicator api: drzave in agregati, indikatorji
%%     - climate api: drzave, tipi podatkov, meritveno obdobje 
%% 
%% - Zahteva za podatke vraca dataset objekt ki ponuja surove podatke, ali pa eno
%%   drugo obliko. 2D array ali pa dict.
%% 
%% - Dataset razred lahko tudi poljubno razsirimo.



\subsection{Pomocnik IndicatorAPI}

IndicatorAPI je razred namenjen pridobivanju podatkov indikatorjev razvoja.
Ker ima programski vmesnik Svetovne banke omejitev koliko podatkov lahko
prenesemo z eno poizvedbo, nam ta razred zdruzuje rezultate vseh poizvedb, ki
so potrebne za prenos celotne zahteve. To poskrbi tako da se po prvi poizvedbi
sprehodi cez stevilo preostalih strani (ref na basic response) ki so na voljo. 

Poleg tega da skrbi za prenos vseh strani podatkov, tudi belezi stevilo 
narejenih in stevilo potrebnih poizvedb za celoten prenos. Do teh stevil lahko
dostopamo z drugih niti in jih uporabimo za prikaz napredka in preostalega
casa do prenosa celotne zahteve.

Glavne metode ki jih ponuja IndicatorAPI razred so:

\begin{itemize}  
\item get\_indicators - vrne seznam vseh moznih indikatorjev z imeni, opisi in
      identifikatorji,
\item get\_countries - vrne seznam drzav in regij z kodami in metapodatki,
\item get\_dataset - vrne razred (IndicatorDataset) ki vsebuje vse podatke z 
      poizvedbe in metode za oblikovanje predstavitve podatkov: api\_responses,
      as\_list, as\_dict.
\end{itemize}




Vrednosti teh podatkov so dolocene z drzavo, casovno komponento, in
identifikatorjem indikatorja. Te podatke lahko predstavimo na dva glavna nacina:

 - kot gnezdeni slovar, kjer je na prvem nivoju ime indikatorja, na drugem
   drzava, in na tretjem nivoju casovna komponenta.

 - Kot dvo-dimenzionalno polje, kjer imamo v vrsticah eno oznako, v stolpcih
   pa kartezicni produkt ostalih dveh. ponujene moznosti so:
   - vrstice = drzava, stolpci = cas x indikator
   - vrstice = cas, stolpci = drzava x indikator


Indicator



% notes:
% 3 dimenzije: cas, drzava, indikator > vrednost
% 
% lahko damo v:
% 
% 
%    \    drzava x indikator
%   cas
% 
%  ali: 
% 
%    \    cas x indikator
% drzava
% 




\subsection{Pomcnik ClimateAPI}

IndicatorAPI je 



% notes:
% 3 dimenzije: cas (decade,year,month), drzava, tip (pr/ tas)
% 
% lahko damo v poljubno konfiguracijo z nastetimi elementi v stolpcu
% 
% stolpci = [drzava, cas] => vrstice = tip ... naredi kartezicni produkt



\section{api\_wrapper}

% doda as orange table in as numpy obema indikator apiju in climate apiju.


razsiritev simple wbd vmesnikov z dedovanjem pravega dataset razreda.

\begin{verbatim}

class ClimateDataset(simple_wbd.ClimateDataset):
    
    def as_numpy(self):
        raise NotImplemented()
    
    def as_orange_table(self):
        raise NotImplemented()

class ClimateAPI(simple_wbd.ClimateAPI):

    def __init__(self):
        super().__init__(ClimateDataset)
\end{verbatim}





\section{Graficni vmesnik}


- Lazja uporaba.
- Vecja preglednost,
- lazje iskanje (text search filter)

\chapter{Primeri uporabe}


\section{Uporaba modula api\_wrapper}

Enostavno uporabo modula \verb|api_wrapper| s skriptnim delom programa Orange
prikazuje primer \ref{scripting_example}. V temu primeru pogledamo, kako
učinkovito lahko napovemo smrtnost otrok iz raznih indikatorjev zdravja,
okolja in infrastrukture. V vrsticah $5$ do $15$ naredimo poizvedbe po
potrebnih podatkih s programskega vmesnika Svetovne banke. Nato v vrsticah $18$
do $27$ odstranimo vrstice iz tabele, ki nimajo ciljne vrednosti in naredimo novo tabelo z
razredom, ki ga želimo napovedovati. Vrednosti, ki jih želimo napovedovati, se
nahajajo v stolpcu $55$ v tabeli \verb|class_data|. Ta stolpec vsebuje podatke
o smrtnosti otrok mlajših od enega leta za leto 2015. V naslednjih vrsticah
pa zgradimo tri napovedne modele: naključni gozd z
regresijskimi drevesi \verb|rf|, linearna regresija z regularizacijo 
\verb|ridge| in srednja vrednost \verb|mean|.
Za ocene napovednih modelov smo uporabili oceni
$RMSE$~\fnurl{https://en.wikipedia.org/wiki/Root-mean-square\_deviation} in 
$R^2$~\fnurl{https://en.wikipedia.org/wiki/Coefficient\_of\_determination}.
Rezultate primera \ref{scripting_example} lahko vidimo v tabeli 
\ref{rezultati_skripte}.


\begin{snippet}
\begin{center}
\lstinputlisting{example.py}
\end{center}
\cprotect
\caption{Napovedovanje smrtnosti otrok do enega leta iz podatkov o dostopnosti
  čiste vode, številu bolniških postelj na 1000 prebivalcev in odstotku
  cepljenih otrok do drugega leta starosti.}
\label{scripting_example}
\end{snippet} 

\begin{table}
\begin{center}

\begin{tabular}{l|r|r}
  Learner & RMSE & R2 \\ \hline
  rf & 9.74 & 0.79 \\
  ridge & 17.76 & 0.31 \\
  mean & 21.35 & -0.00
\end{tabular}
\end{center}
\cprotect
\caption{Rezultati napovedi smrtnosti otrok do enega leta starosti.}
\label{rezultati_skripte}
\end{table} 



\section{Napoved temperature s pomočjo $CO_2$ izpustov v ZDA}


Podatke svetovne banke lahko uporabimo tudi kot časovne vrste z uporabo
posebnih gradnikov za delo s časovnimi vrstami \cite{time_series}. Tukaj si
bomo ogledali enostaven primer napovedi temperature v ZDA s pomočjo podatkov o
izpustih $CO_2$. V tej napovedi smo uporabili podatke tako z gradnika 
WB Indicators (Slika \ref{var_indicator_select})
kot tudi z gradnika WB Climate (Slika \ref{var_climate_select}). Podatke obeh
gradnikov smo zdru"zili z gradnikom ``Merge Data'' po obeh "casovnih
komponentah. Nato smo odstranili vnose "casovnih obdobij za katere nimamo na
voljo vseh podatkov. Sestavljeno tabelo prikazuje slika \ref{var_data_table}.
Iz teh podatkov nato zgradimo "casovno vrsto in s pomočjo modela vektorske 
autoregresije VAR \cite{var_model} napovemo podatke za povprečno 
letno temperaturo za naslednjih nekaj let, kar je prikazano na sliki 
\ref{var_forecast_graph}.

\begin{figure}
\begin{center}
\includegraphics[width=13.75cm]{pic/var_setup.png}
\end{center}
\caption{Prikaz povezave gradnikov za napoved temperature.}
\label{var_setup}
\end{figure} 


\begin{figure}
\begin{center}
\includegraphics[width=13.75cm]{pic/var_indicator_select.png}
\end{center}
\caption{Izbor indikatorja $CO_2$ izpustov v ZDA.}
\label{var_indicator_select}
\end{figure} 

\begin{figure}
\begin{center}
\includegraphics[width=8cm]{pic/var_climate_select.png}
\end{center}
\caption{Izbor podatkov povprečnih letnih temperatur v ZDA.}
\label{var_climate_select}
\end{figure} 

\begin{figure}
\begin{center}
\includegraphics[width=10cm]{pic/var_data_table.png}
\end{center}
\caption{Podatkovna tabela s ciljnim razredom, in dvema poljema.}
\label{var_data_table}
\end{figure} 

\begin{figure}
\begin{center}
\includegraphics[width=13.75cm]{pic/var_forecast_graph.png}
\end{center}
\caption{Prikaz napovedi gibanja povprečnih letnih temperatur ``USA - tas'' in
  $CO_2$ izpustov ``United States''.}
\label{var_forecast_graph}
\end{figure} 




\section{Gru"cenje dr"zav}


Podatke, ki jih dobimo z na"sim dodatkom, lahko v programu Orange uporabimo tudi
za grafi"cni prikaz statistik in povezav med dr"zavami. Kot mo"zen primer
uporabe (Slika \ref{clustering_setup}) smo prikazali gru"cenje dr"zav svetovnih regij glede na naslednje
indikatorje (Slika \ref{clustering_indicator_selection}):
\begin{itemize}
  \item odstotek ljudi ki "zivijo v urbanem okolju 
    \angl{Urban population (\% of total)},
  \item smrtnost na $1000$ "zivorojenih otrok
    \angl{Mortality rate, infant (per 1,000 live births)},
  \item "stevilo bolni"skih postelj na $1000$ prebivalcev
    \angl{Hospital beds (per 1,000 people)},
  \item odstotek BDP izdatkov za raziskave in razvoj
    \angl{Research and development expenditure (\% of GDP)},
  \item "stevilo prebivalstva pod pragom rev"s"cine pri meji $3.10$ dolarjev na dan
    \angl{Poverty gap at $\$3.10$ a day (2011 PPP) (\%)}.
\end{itemize}
Med temi indikatorji smo izra"cunali evklidsko razdaljo in za prikaz 
uporabili "ze obstoje"ca gradnika programa Orange
``MDS'' (slika \ref{clustering_mds}) in
``Hierarchical Clustering'' (slika \ref{clustering_hierarchial_countries}).



\angl{multidimensional scaling}. 
% ID: GB.XPD.RSDV.GD.ZS 
% ID: SH.MED.BEDS.ZS 
% ID: SI.POV.GAP2 
% ID: SP.DYN.IMRT.IN 
% ID: SP.URB.TOTL.IN.ZS 


\begin{figure}
\begin{center}
\includegraphics[width=5cm]{pic/clustering_setup.png}
\end{center}
\caption{Postavitev okolja za prikaz gru"cenja.}
\label{clustering_setup}
\end{figure} 

\begin{figure}
\begin{center}
\includegraphics[width=13.75cm]{pic/clustering_indicator_selection.png}
\end{center}
\caption{Izbor indikatorjev za gru"cenje.}
\label{clustering_indicator_selection}
\end{figure} 

\begin{figure}
\begin{center}
\includegraphics[width=13.75cm]{pic/clustering_hierarchial_countries.png}
\end{center}
\caption{Prikaz hierarhi"cnega gru"cenja dr"zav.}
\label{clustering_hierarchial_countries}
\end{figure} 

\begin{figure}
\begin{center}
\includegraphics[width=13.75cm]{pic/clustering_mds.png}
\end{center}
\caption{Prikaz gru"cenja MDS.}
\label{clustering_mds}
\end{figure} 


\chapter{Sklepne ugotovitve}


prikazuje slika \ref{drevo}. 

\begin{figure}
\begin{center}
\includegraphics[width=10cm]{pic/T2FnHj6.png}
\end{center}
\caption{Pinkie Pie}
\label{drevo}
\end{figure} 


\begin{thebibliography}{1}

% Indicator data sources

\bibitem{world_dev_ind} World Development Indicators, The World Bank, (August 2016)
\\ URL: \url{http://data.worldbank.org/data-catalog/world-development-indicators}

\bibitem{glob_fin_dev} Data source: Global Financial Development Database (GFDD), The World Bank. Methodology citation: Martin Čihák, Aslı Demirgüç-Kunt, Erik Feyen, and Ross Levine, 2012. "Benchmarking Financial Systems Around the World." World Bank Policy Research Working Paper 6175, World Bank, Washington, D.C. (Junij 2016)
\\ \url{http://data.worldbank.org/data-catalog/global-financial-development}

% \bibitem{int_debt_stat} International Debt Statistics, The World Bank (December 2015) 
% \\ \url{http://data.worldbank.org/data-catalog/international-debt-statistics}

\bibitem{africa_dev_ind} Africa Development Indicators, The World Bank (Februar 2013)
\\ \url{http://data.worldbank.org/data-catalog/africa-development-indicators}

\bibitem{doing_buseness} Doing Business, The World Bank (http://www.doingbusiness.org) (Julij 2016)
\\ \url{http://data.worldbank.org/data-catalog/doing-business-database}

\bibitem{ent_surveys} Enterprise Surveys, The World Bank (Julij 2016)
\\ \url{http://data.worldbank.org/data-catalog/enterprise-surveys}

\bibitem{mil_dev_goals} Millennium Development Goals, The World Bank (Julij 2016)
\\ \url{http://data.worldbank.org/data-catalog/millennium-development-indicators}

\bibitem{edu_stat} World Bank EdStats (Junij 2016)
\\ \url{http://data.worldbank.org/data-catalog/ed-stats}

\bibitem{gen_stat} Gender Statistics, The World Bank (Julij 2016)
\\ \url{http://data.worldbank.org/data-catalog/gender-statistics}

\bibitem{health_pop_stat} HealthStats, World Bank Group (Julij 2016)
\\ \url{http://data.worldbank.org/data-catalog/health-nutrition-and-population-statistics}

\bibitem{ida_res_mes_sys} IDA Results Measurement System, the World Bank (Julij 2016)
\\ \url{http://data.worldbank.org/data-catalog/IDA-results-measurement}

\bibitem{climate_data} Climatic Research Unit, University of East Anglia
\\ \url{http://www.cru.uea.ac.uk/data}

\bibitem{orange_scripting} Janez Dem"sar and Toma"z Curk and Ale"s Erjavec and "Crt Gorup and Toma"z Ho"cevar and Mitar Milutinovi"c and Martin Mo"zina and Matija Polajnar and Marko Toplak and An"ze Stari"c and Miha "Stajdohar and Lan Umek and Lan "Zagar and Jure "Zbontar and Marinka "Zitnik and Bla"z Zupan, ``Orange: Data Mining Toolbox in Python,'' Journal of Machine Learning Research, vol. 14, pp. 2349-2353, 2013.


\bibitem{time_series} Jernej Kernc, ``Orodje za interaktivno analizo časovnih vrst,'' 2016





\bibitem{jezicno} Jure Dimec (2002), Medjezično iskanje dokumentov 
\\ \url{http://clir.craynaud.com/clir/MEDJEZICNOISKANJEDOKUMENTOV.pdf}


% \bibitem{opengl} (Avgust, 2013) OpenGL Overview
% \\ \url{http://www.opengl.org/about/}

\end{thebibliography}


%\begin{thebibliography}{99}
%\bibitem{lf} L.\ Fortnow, ``Viewpoint: Time for computer science to grow up'',
%{\it Communications of the ACM}, št.\ 52, zv.\ 8, str.\ 33--35, 2009.
%\bibitem{dk1} D.\ E.\ Knuth, P. Bendix. ``Simple word problems in universal algebras'', v zborniku: Computational Problems in Abstract Algebra (ur. J. Leech), 1970, str. 263--297.
%\bibitem{lat} L.\ Lamport. {\it LaTEX: A Document Preparation System}. Addison-Wesley, 1986.
%\bibitem{bib} O.\ Patashnik (1998) \BibTeX{}ing. 
%\\ http://ftp.univie.ac.at/packages/tex/biblio/bibtex/contrib/doc/btxdoc.pdf
%\bibitem{licence} licence-cc.pdf. Dostopno na: 
%\end{thebibliography}


\end{document}

