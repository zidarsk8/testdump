%%%%%%%%%%%%%%%%%%%%%%%%%%%%%%%%%%%%%%%%
% datoteka diploma-vzorec.tex
%
% vzorčna datoteka za pisanje diplomskega dela v formatu LaTeX
% na UL Fakulteti za računalništvo in informatiko
%
% vkup spravil Gašper Fijavž, december 2010
% 
%
%
% verzija 12. februar 2014 (besedilo teme, seznam kratic, popravki Gašper Fijavž)
% verzija 10. marec 2014 (redakcijski popravki Zoran Bosnić)
% verzija 11. marec 2014 (redakcijski popravki Gašper Fijavž)
% verzija 15. april 2014 (pdf/a 1b compliance, not really - just claiming, Damjan Cvetan, Gašper Fijavž)
% verzija 23. april 2014 (privzeto cc licenca)
% verzija 16. september 2014 (odmiki strain od roba)
% verzija 28. oktober 2014 (odstranil vpisno številko)
% verija 5. februar 2015 (Literatura v kazalu, online literatura)
% verzija 25. september 2015 (angl. naslov v izjavi o avtorstvu)
% verzija 26. februar 2016 (UL izjava o avtorstvu)
% verzija 16. april 2016 (odstranjena izjava o avtorstvu)


\documentclass[a4paper, 12pt]{book}

\usepackage[utf8x]{inputenc}   % omogoča uporabo slovenskih črk kodiranih v formatu UTF-8
\usepackage[slovene,english]{babel}    % naloži, med drugim, slovenske delilne vzorce
\usepackage[pdftex]{graphicx}  % omogoča vlaganje slik različnih formatov
\usepackage{fancyhdr}          % poskrbi, na primer, za glave strani
\usepackage{amssymb}           % dodatni simboli
\usepackage{amsmath}           % eqref, npr.
%\usepackage{hyperxmp}
\usepackage[pdftex, colorlinks=true,
						citecolor=black, filecolor=black, 
						linkcolor=black, urlcolor=black,
						pagebackref=false, 
						pdfproducer={LaTeX}, pdfcreator={LaTeX}, hidelinks]{hyperref}

\usepackage{microtype}%za lepso postavitev/delitev besed
\usepackage{cprotect}
\usepackage{listingsutf8}
\lstset{
    % osnovno
    tabsize=4,
    basicstyle={\ttfamily \scriptsize},
    identifierstyle={\ttfamily},
    inputencoding=cp1250,
    %
    % barvanje kode
    commentstyle={\sffamily \color[rgb]{0,0.5,0}},
    stringstyle=\color[rgb]{0.5,0,1},
    keywordstyle=\color[rgb]{0,0,1},
    language=Python,
    showstringspaces=false,
    showspaces=false,
    %
    % iz source datotek potegnemo tisto med //@begin@ ...koda... //@end@
    % rangeprefix={\#@},
    % rangesuffix={@},
    % linerange=begin-end,
    % includerangemarker=false,
    %
    % lomljenje predolgih vrstic
    breaklines=true,
    prebreak=\raisebox{0ex}[0ex][0ex]{\ensuremath{\hookleftarrow}},
    %
    % okvir
    frame=single,
    frameround={t}{t}{t}{t},
    xleftmargin=23pt,
    xrightmargin=4pt,
    framexleftmargin=19pt,
    framerule=0pt,
    %
    % številčenje
    numbers=left,
    firstnumber=1,
    extendedchars=true,
}

%%%%%%%%%%%%%%%%%%%%%%%%%%%%%%%%%%%%%%%%
%	DIPLOMA INFO
%%%%%%%%%%%%%%%%%%%%%%%%%%%%%%%%%%%%%%%%
\newcommand{\ttitle}{Dostop do podatkov svetovne banke v orodju Orange}
\newcommand{\ttitleEn}{The World Bank Data Access from Orange Data Mining Toolkit}
\newcommand{\tsubject}{\ttitle}
\newcommand{\tsubjectEn}{\ttitleEn}
\newcommand{\tauthor}{Miha Zidar}
\newcommand{\tkeywords}{podatkovno rudarjenje, programski vmesnik, Svetovna banka, gospodarski indikatorji, podnebni podatki, Orange}
\newcommand{\tkeywordsEn}{Data mining, API, World Bank, indicators, climate, Orange}



\usepackage{hyperref}
%%%%%%%%%%%%%%%%%%%%%%%%%%%%%%%%%%%%%%%%
%	HYPERREF SETUP
%%%%%%%%%%%%%%%%%%%%%%%%%%%%%%%%%%%%%%%%
\hypersetup{pdftitle={\ttitle}}
\hypersetup{pdfsubject=\ttitleEn}
\hypersetup{pdfauthor={\tauthor, miha@zidar.me}}
\hypersetup{pdfkeywords=\tkeywordsEn}

\usepackage{emptypage}
\usepackage{color}
\usepackage{float}
\newfloat{snippet}{thp}{lop}
\floatname{snippet}{Primer}

\newcommand\angl[1]{({angl.\ }{\it#1})}

\newcommand\fnurl[1]{%
    \footnote{\url{#1}}%
}

%%%%%%%%%%%%%%%%%%%%%%%%%%%%%%%%%%%%%%%%
% postavitev strani
%%%%%%%%%%%%%%%%%%%%%%%%%%%%%%%%%%%%%%%%  

\addtolength{\marginparwidth}{-20pt} % robovi za tisk
\addtolength{\oddsidemargin}{40pt}
\addtolength{\evensidemargin}{-40pt}

\renewcommand{\baselinestretch}{1.3} % ustrezen razmik med vrsticami
\setlength{\headheight}{15pt}        % potreben prostor na vrhu
\renewcommand{\chaptermark}[1]%
{\markboth{\MakeUppercase{\thechapter.\ #1}}{}} \renewcommand{\sectionmark}[1]%
{\markright{\MakeUppercase{\thesection.\ #1}}} \renewcommand{\headrulewidth}{0.5pt} \renewcommand{\footrulewidth}{0pt}
\fancyhf{}
\fancyhead[LE,RO]{\sl \thepage} \fancyhead[LO]{\sl \rightmark} \fancyhead[RE]{\sl \leftmark}



\newcommand{\BibTeX}{{\sc Bib}\TeX}

%%%%%%%%%%%%%%%%%%%%%%%%%%%%%%%%%%%%%%%%
% naslovi
%%%%%%%%%%%%%%%%%%%%%%%%%%%%%%%%%%%%%%%%  


\newcommand{\autfont}{\Large}
\newcommand{\titfont}{\LARGE\bf}
\newcommand{\clearemptydoublepage}{\newpage{\pagestyle{empty}\cleardoublepage}}
\setcounter{tocdepth}{1}	      % globina kazala

%%%%%%%%%%%%%%%%%%%%%%%%%%%%%%%%%%%%%%%%
% konstrukti
%%%%%%%%%%%%%%%%%%%%%%%%%%%%%%%%%%%%%%%%  
\newtheorem{izrek}{Izrek}[chapter]
\newtheorem{trditev}{Trditev}[izrek]
\newenvironment{dokaz}{\emph{Dokaz.}\ }{\hspace{\fill}{$\Box$}}

%%%%%%%%%%%%%%%%%%%%%%%%%%%%%%%%%%%%%%%%%%%%%%%%%%%%%%%%%%%%%%%%%%%%%%%%%%%%%%%
%% PDF-A
%%%%%%%%%%%%%%%%%%%%%%%%%%%%%%%%%%%%%%%%%%%%%%%%%%%%%%%%%%%%%%%%%%%%%%%%%%%%%%%

%%%%%%%%%%%%%%%%%%%%%%%%%%%%%%%%%%%%%%%% 
% define medatata
%%%%%%%%%%%%%%%%%%%%%%%%%%%%%%%%%%%%%%%% 
\def\Title{\ttitle}
\def\Author{\tauthor, matjaz.kralj@fri.uni-lj.si}
\def\Subject{\ttitleEn}
\def\Keywords{\tkeywordsEn}

%%%%%%%%%%%%%%%%%%%%%%%%%%%%%%%%%%%%%%%% 
% \convertDate converts D:20080419103507+02'00' to 2008-04-19T10:35:07+02:00
%%%%%%%%%%%%%%%%%%%%%%%%%%%%%%%%%%%%%%%% 
\def\convertDate{%
    \getYear
}

{\catcode`\D=12
 \gdef\getYear D:#1#2#3#4{\edef\xYear{#1#2#3#4}\getMonth}
}
\def\getMonth#1#2{\edef\xMonth{#1#2}\getDay}
\def\getDay#1#2{\edef\xDay{#1#2}\getHour}
\def\getHour#1#2{\edef\xHour{#1#2}\getMin}
\def\getMin#1#2{\edef\xMin{#1#2}\getSec}
\def\getSec#1#2{\edef\xSec{#1#2}\getTZh}
\def\getTZh +#1#2{\edef\xTZh{#1#2}\getTZm}
\def\getTZm '#1#2'{%
    \edef\xTZm{#1#2}%
    \edef\convDate{\xYear-\xMonth-\xDay T\xHour:\xMin:\xSec+\xTZh:\xTZm}%
}

\expandafter\convertDate\pdfcreationdate 

%%%%%%%%%%%%%%%%%%%%%%%%%%%%%%%%%%%%%%%%
% get pdftex version string
%%%%%%%%%%%%%%%%%%%%%%%%%%%%%%%%%%%%%%%% 
\newcount\countA
\countA=\pdftexversion
\advance \countA by -100
\def\pdftexVersionStr{pdfTeX-1.\the\countA.\pdftexrevision}


%%%%%%%%%%%%%%%%%%%%%%%%%%%%%%%%%%%%%%%%
% XMP data
%%%%%%%%%%%%%%%%%%%%%%%%%%%%%%%%%%%%%%%%  
% \usepackage{xmpincl}
% \includexmp{pdfa-1b}

%%%%%%%%%%%%%%%%%%%%%%%%%%%%%%%%%%%%%%%%
% pdfInfo
%%%%%%%%%%%%%%%%%%%%%%%%%%%%%%%%%%%%%%%%  
\pdfinfo{%
    /Title    (\ttitle)
    /Author   (\tauthor)
    /Subject  (\ttitleEn)
    /Keywords (\tkeywordsEn)
    /ModDate  (\pdfcreationdate)
    /Trapped  /False
}


%%%%%%%%%%%%%%%%%%%%%%%%%%%%%%%%%%%%%%%%%%%%%%%%%%%%%%%%%%%%%%%%%%%%%%%%%%%%%%%
%%%%%%%%%%%%%%%%%%%%%%%%%%%%%%%%%%%%%%%%%%%%%%%%%%%%%%%%%%%%%%%%%%%%%%%%%%%%%%%

\begin{document}
\selectlanguage{slovene}
\frontmatter
\setcounter{page}{1} %
\renewcommand{\thepage}{}       % preprecimo težave s številkami strani v kazalu

%%%%%%%%%%%%%%%%%%%%%%%%%%%%%%%%%%%%%%%%
%naslovnica
 \thispagestyle{empty}%
   \begin{center}
    {\large\sc Univerza v Ljubljani\\%
      Fakulteta za računalništvo in informatiko}%
    \vskip 10em%
    {\autfont Miha Zidar \par}%
    {\titfont Dostop do podatkov Svetovne banke v orodju Orange \par}%
    {\vskip 2em \textsc{DIPLOMSKO DELO\\[2mm] 
    UNIVERZITETNI ŠTUDIJSKI PROGRAM RAČUNALNIŠTVO IN INFORMATIKA}\par}%
    \vfill\null%
    {\large \textsc{Mentor}: prof. dr. Blaž Zupan \par}%
    {\vskip 2em \large Ljubljana, 2016 \par}%
\end{center}
% prazna stran
\clearemptydoublepage

%%%%%%%%%%%%%%%%%%%%%%%%%%%%%%%%%%%%%%%%
%copyright stran
\thispagestyle{empty}
\vspace*{8cm}
Fakulteta za računalništvo in informatiko podpira javno dostopnost znanstvenih, strokovnih in razvojnih rezultatov. Zato priporoča objavo dela pod katero od licenc, ki omogočajo prosto razširjanje diplomskega dela in/ali možnost nadaljne proste uporabe dela. Ena izmed možnosti je izdaja diplomskega dela pod katero od Creative Commons licenc \href{http://creativecommons.si}{http://creativecommons.si}

Morebitno pripadajočo programsko kodo praviloma objavite pod, denimo, licenco 
\emph{GNU General Public License, različica 3}. Podrobnosti licence so dostopne na spletni strani \href{http://www.gnu.org/licenses/}{http://www.gnu.org/licenses/}.

\begin{center}
\mbox{}\vfill
\emph{Besedilo je oblikovano z urejevalnikom besedil \LaTeX.}
\end{center}
% prazna stran
\clearemptydoublepage

%%%%%%%%%%%%%%%%%%%%%%%%%%%%%%%%%%%%%%%%
% stran 3 med uvodnimi listi
\thispagestyle{empty}
\vspace*{4cm}

\noindent
Fakulteta za računalništvo in informatiko izdaja naslednjo nalogo:
\medskip
\begin{tabbing}
\hspace{32mm}\= \hspace{6cm} \= \kill




Tematika naloge:
\end{tabbing}
Na spletu je veliko odprtih baz podatkov, za katere bi bilo koristno, da bi bile dostopne v orodjih za podatkovno analitiko. Primer take baze so spletne strani in programski vmesniki Svetovne banke, preko katerih lahko dostopamo do demografskih, gospodarskih in klimatskih podatkov. V nalogi razvijte knjižnico in komponente z grafičnimi vmesnikom za program Orange, s katerimi lahko enostavno in hitro pridobimo podatke iz Svetovne banke in jih z obstoječimi analitičnimi gradniki v Orange-u tudi analiziramo.
\vspace{15mm}






\vspace{2cm}

% prazna stran
\clearemptydoublepage

% zahvala
\thispagestyle{empty}\mbox{}\vfill\null\it%
Zahvalil bi se mentorju, prof.\ dr.\ Blažu Zupanu in članom laboratorija za 
bioinformatiko za pomoč in usmerjanje med izdelavo diplomskega dela. 
Prav tako bi se zahvalil svojemu partnerju, staršem in prijateljem za
spodbudo.
\rm\normalfont

% prazna stran
\clearemptydoublepage

%%%%%%%%%%%%%%%%%%%%%%%%%%%%%%%%%%%%%%%%%
%% posvetilo
%\thispagestyle{empty}\mbox{}{\vskip0.20\textheight}\mbox{}\hfill\begin{minipage}{0.55\textwidth}%
%% Za bubija
%\normalfont\end{minipage}
%
%% prazna stran
%\clearemptydoublepage

%%%%%%%%%%%%%%%%%%%%%%%%%%%%%%%%%%%%%%%%
% kazalo
\pagestyle{empty}
\def\thepage{}% preprecimo tezave s stevilkami strani v kazalu
\tableofcontents{}


% prazna stran
\clearemptydoublepage

%%%%%%%%%%%%%%%%%%%%%%%%%%%%%%%%%%%%%%%%
% seznam kratic

% \chapter*{Seznam uporabljenih kratic}
% 
% \begin{tabular}{l|l|l}
%   {\bf kratica} & {\bf angleško} & {\bf slovensko} \\ \hline
%   % after \\: \hline or \cline{col1-col2} \cline{col3-col4} ...
%   {\bf CA} & classification accuracy & klasifikacijska točnost \\
%   {\bf DBMS} & database management system & sistem za upravljanje podatkovnih baz \\
%   {\bf SVM} & support vector machine & metoda podpornih vektorjev \\
%   \dots & \dots & \dots \\
% \end{tabular}



% prazna stran
\clearemptydoublepage

%%%%%%%%%%%%%%%%%%%%%%%%%%%%%%%%%%%%%%%%
% povzetek
%%%%%%%%%%%%%%%%%%%%%%%%%%%%%%%%%%%%%%%%
% povzetek 
\addcontentsline{toc}{chapter}{Povzetek}
\chapter*{Povzetek}


\textbf{Naslov:} Dostop do podatkov Svetovne banke v orodju Orange

\ \\
\textbf{Avtor:} Miha Zidar

\ \\
% TODO dodaj kaksen stavek ali dva
Program Orange je orodje za podatkovno rudarjenje, v katerem
lahko za namene analiz uporabimo različne podatkovne vire. Sam program Orange
vsebuje predpripravljene zbirke podatkov, dodatne zbirke podatkov si lahko 
pripravi in uvozi tudi uporabnik sam, ali pa uporabi katerega od že obstoječih
dodatkov za uvoz podatkov. Za namen diplomske naloge smo izdelali dodatek imenovan
Orange Data Sets, ter v njem razvili gradnike za dostop do podatkov s programskega 
vmesnika Svetovne banke. Svetovna banka omogoča uporabo štirih 
različnih programskih vmesnikov: gospodarski indikatorji, finančni podatki,
projekti Svetovne banke in podnebni podatki. V dodateku Orange Data Sets smo
razvili gradnike za branje in uporabo podatkov indikatorjev 
in podnebnih podatkov.
S tem bo uporabnikom programa Orange omogočena enostavnejša uporaba velikega števila
podatkov iz omenjenih dveh programskih vmesnikov.

\ \\
\textbf{Ključne besede:} podatkovno rudarjenje, programski vmesnik, 
Svetovna banka, gospodarski indikatorji, podnebni podatki, Orange. 




% prazna stran
\clearemptydoublepage

%%%%%%%%%%%%%%%%%%%%%%%%%%%%%%%%%%%%%%%%
% abstract
\selectlanguage{english}
\addcontentsline{toc}{chapter}{Abstract}
\chapter*{Abstract}


\textbf{Title:} Access to World Bank Data with Orange

\ \\
\textbf{Author:} Miha Zidar

\ \\
Orange is an open source data-mining software, capable of using multiple sources for data analysis. There are a few test data samples already present in Orange, and the user can import their own data sets with the use of one of Orange input widgets. For this thesis, we created an add-on Orange Data Sets, that includes widgets for accessing free data from World Bank application program interface (API). The World Bank exposes four different data APIs; indicator, project, finance and climate. Widgets included in the Orange Data Sets add-on are be able to read data from the World Bank Indicators and World Bank Climate APIs.
This will enable Orange users a quick and easy access to data from the two previously mentioned APIs.

\ \\
\textbf{Key words:} Data mining, API, World Bank, indicators, climate, Orange.

\selectlanguage{slovene}


\selectlanguage{slovene}
% prazna stran
\clearemptydoublepage

%%%%%%%%%%%%%%%%%%%%%%%%%%%%%%%%%%%%%%%%
\mainmatter
\setcounter{page}{1}
\pagestyle{fancy}



\chapter{Uvod}

Na svetovnem spletu je dosegljivih vedno več prosto dostopnih programskih
vmesnikov \angl{application programming interface}.
Ti vmesniki omogočajo dostop
do zelo raznolikih zbirk podatkov. Primeri takih zbirk so
seznam stopnje ogroženosti živali po državah\fnurl{http://apiv3.iucnredlist.org/api/v3/docs},
podatki meritev in slike vesolja agencije NASA\fnurl{https://api.nasa.gov/},
seznam knjig z ocenami in povezavami med uporabniki\fnurl{https://www.goodreads.com/api},
zgodovina meteoroloških meritev\fnurl{http://climatedataapi.worldbank.org/},
razni indikatorji stopenj razvoja držav\fnurl{http://api.worldbank.org/}.
    

Programski vmesniki so oblikovani tako, da je omogočena raznolika uporaba
vmesnika, s katerim dolo"cimo, katere podatke "zelimo pridobiti.
Ta fleksibilnost pa ima tudi slabost, saj je 
podatke potrebno predhodno obdelati za vsak namen posebej. Tako bi na primer 
moral vsak uporabnik programa Orange, sicer splo"sno uporabnega okolja za 
podatkovno analitiko, podatke predhodno pretvoriti v obliko, 
primerno za dani problem in cilje zastavljene analize.


\section{Motivacija}

Povezava programskega vmesnika za dostop do podatkov in orodja za analizo 
podatkov je pogosto prezapletena za končnega uporabnika. Razviti želimo
knjižnice in dodatka za program Orange, s katerimi bi podatke s programskega
vmesnika Svetovne banke pripravili v obliki primerni za nadaljnjo
uporabo v orodju Orange in drugih programih za obdelavo podatkov. S tem bi
dobili enostavnejši dostop do preko 16.000 indikatorjev in številnih podnebnih
meritev, s čimer bomo lažje analizirali in iskali morebitne zakonitosti v
podatkih. Če bi imeli en sam ustrezen dodatek za dostop do podatkov
programskega vmesnika Svetovne banke, bi se poenostavilo tudi posodabljanje in
vzdrževanje kode v primeru sprememb programskega vmesnika. S tem odpravimo
potrebo, da bi moral vsak uporabnik sam skrbeti za uskladitvene posodobitve,
ampak se vmesnik posodobi enkrat in za vse uporabnike.


\section{Cilji in struktura diplomske naloge}

Cilj diplomske naloge je izdelati knjižnico za uporabo programskega vmesnika
Svetovne banke za programski jezik Python ter izdelati dodatek za program Orange, ki s pomočjo omenjene
knjižnice omogoča uporabniku dostop do podatkov Svetovne banke preko
grafičnega vmesnika.


V diplomski nalogi najprej predstavimo spletna vira indikatorjev
držav sveta in meritev podnebnih podatkov Svetovne banke ter
opišemo delovanje njunih programskih vmesnikov.
Nato podrobneje opišemo našo implementacijo knjižnice za dostop do
programskega vmesnika Svetovne banke in gradnikov za program Orange, ki to
knjižnico uporabljajo. V nadaljevanju prikažemo še nekaj praktičnih 
primerov uporabe razvitih gradnikov. Na koncu še popišemo opravljeno 
delo, navedemo vire programske kode in omenimo možne načine za izboljšavo ali 
nadgradnjo našega dodatka.












% bomo podrobneje opisali programski vmesnik za dostop do
% podatkov Svetovne banke (API SB). V četrtem poglavju sledi predstavitev
% knjižnice in gradnikov za Orange in nato še konkretni primeri uporabe. Na koncu
% bomo popisali opravljeno delo, navedli vire kode in opisali nadaljne možnosti
% nadgradnje dodatka.


%% -------------------------------
%%
%%  - Prav tako se večina programov in knjižnic za dostop do baz podatkov osredotoči le na iskanje po teh bazah, ne pa tudi na pridobivanje čim večje količine podatkov.
%%
%%
%% % Poleg tega obstaja dosti odprto kodnih programov za obdelavo
%% % in analizo podatkov. Ker so programski vmesnike bolj splošno namenski, je
%%
%%
%% \chapter{Uvod}
%%
%%
%% Na spletu je vedno več prosto dostopnih programskih vmesnikov(API, ang.
%% application programming interface) za različne baze podatkov. Ti vmesniki
%% ponujajo dostop do zelo raznolikih podatkov, kot so
%% seznami stopnje ogroženosti živali po državah
%% \fnurl{http://apiv3.iucnredlist.org/api/v3/docs},
%% NASA podatki meritev in slike vesolja
%% \fnurl{https://api.nasa.gov/},
%% seznam knjig z ocenami in povezavami med uporabniki
%% \fnurl{https://www.goodreads.com/api},
%% zgodovina meteoroloških meritev
%% \fnurl{http://climatedataapi.worldbank.org/},
%% razni indikatorji stopnje razvoja držav
%% \fnurl{http://api.worldbank.org/}.
%%
%% Ker pa so programski vmesniki bolj splošno namenski, je
%% podatke teško spraviti v obliko ki bi bila primerna za uporabo v raznih
%% orodjih za analizo in obdelavo podatkov. Prav tako se večina programov in knjižnic za dostop
%% do baz podatkov osredotoči le na iskanje po teh bazah, ne pa tudi na
%% pridobivanje čim večje količine podatkov.
%%
%%
%% % Poleg tega obstaja dosti odprto kodnih programov za obdelavo
%% % in analizo podatkov. Ker so programski vmesnike bolj splošno namenski, je
%% % podatke teško spraviti v obliko za analizo in obdelavo. Z orodjem, ki bi
%% % pomagalo združiti programe za obdelavo podatkov in prosto dostopne baze
%% % podatkov, bi omogočili raziskovanje teh podatkov širši javnosti.
%%
%%
%%
%% % Povezava programskega vmesnika in orodja za analizo podatkov pa je pogosto
%% % prezapletena za navadnega uporabnika. Z dodajanjem gradnikov za enostavno
%% % uporabo spletnih programskih vmesnikov v orodjih kot je Orange, omogočimo ...
%%
%% \section{Motivacija}
%%
%% Branje podatkov z raznih programskih vmesnikov je lahko zelo zamudno delo.
%% Programski vmesnik se lahko s časom spremeni, in podatki ki jih dobimo z
%% vmesnika so lahko pokvarjeni. Trenutni pristop, kjer moramo podatke vsakič ročno
%% obilkovati da so primerni za analizo, ima mnogo pomanjkljivosti. Prejeti podatki
%% lahko vsebujejo nepravilnosti, ali pa so celo nedostopni. Z dodatkom ki bi
%% poskrbel za prenos podatkov in pretovrbo v uporabno obliko, hkrati pa bi znal
%% popraviti ali odstraniti pokvarjene podatke, bi lahko več pozornosti posvetili
%% sami analizi in obdelavi. Poleg tega, pa večina knjižnic za delo z odprtimi
%% programskimi vmesniki, nudi zelo dobre načine za iskanje posameznih podatkov,
%% ne pa za prenos večje količine podatkov kar je bolj primerno za analizo in
%% obdelavo.
%%
%%
%%
%% \section{Cilji}
%%
%% Cilj diplomske naloge je izdelati knjižnico, ki omogoča enostaven dostop do
%% podatkov Svetovne banke in interaktivni gradnik v programu Orange za dostop in
%% uporabo teh podatkov. S tem bomo omogočili raziskovanje teh podtakov širši
%% javnosti. Knjižnica bo poenostavila prenos večjega števila podatkov, in
%% predstavila te podatke v obliki primerni za orodje Orange. k<

\chapter{Podatkovne zbirke Svetovne banke}

Pri diplomski nalogi smo se osredotočili na dva programska vmesnika za dostop 
podatkov Svetovne banke. To sta ``ClimateAPI'', s katerim dostopamo do 
podatkovne zbirke meteoroloških meritev in ``IndicatorAPI'', s katerim dostopamo do 
zbirke podatkov raznih indikatorjev stopenj razvoja držav.
Za uporabo podatkovne zbirke Svetovne banke smo se odločili, ker združuje in na
enovit način predstavi podatke iz več različnih virov. Podatkovni viri za 
indikatorje stopnje razvoja držav so:
\begin{itemize}  
  \item Svetovni indikatorji razvoja~\cite{world_dev_ind}, % World Development Indicators, 
  \item Globalni finančni razvoj~\cite{glob_fin_dev},
  \item Afriški indikatorji razvoja~\cite{africa_dev_ind},
  \item Poslovanje~\cite{doing_buseness},
  \item Podjetniške raziskave~\cite{ent_surveys}, 
  \item Razvojni cilji~\cite{mil_dev_goals}, 
  \item Statistike izobraževanja~\cite{edu_stat}, 
  \item Statistike spolov~\cite{gen_stat},
  \item Statistike zdravja in prehranjevanja~\cite{health_pop_stat} in
  \item Rezultati meritev IDA~\cite{ida_res_mes_sys}.
\end{itemize}  

Podatkovni vir zbirke podnebnih meritev pa je osnovan na podatkih oddelka
za podnebne raziskave \angl{Climatic Research Unit}~\cite{climate_data}.

Svetovna banka omogoča dostop do podatkov preko programskega vmesnika 
predstavitvene arhitekture za prenos podatkov REST 
\angl{Representational State Transfer}, ki
ponuja veliko možnosti za iskanje in izbor rezultatov programskih poizvedb. Pri vsaki 
poizvedbi REST lahko določimo želeno obliko odgovora. Za poizvedbe o 
informacijah indikatorjev sta na voljo obliki 
razširljivega označevalnenga jezika XML \angl{Extensible Markup Language} 
in javascript objektne notacije JSON \angl{JavaScript Object Notation}. Programski vmesnik 
meteoroloških meritev pa ponuja samo obliko JSON. Za konsistentnost in lažjo
berljivost smo na obeh programskih vmesnikih uporabili obliko JSON. To na
programskem vmesniku indikatorjev dosežemo tako da nastavimo parameter GET
\verb|format| na vrednost \verb|json|. 


\section{Podatki indikatorjev razvoja držav}
\label{sec:podatki_ind_razvoja}



Programski vmesnik indikatorjev razvoja držav Svetovne banke omogoča dostop
do podatkov preko 16.000 raznih indikatorjev. Podatki indikatorjev so merjeni v
mesečnem, četrtletnem ali letnem intervalu. Začetek meritev podatkov
posameznega indikatorja je odvisna od vira podatkov. Najstarejši podatki segajo
do leta 1960. Poleg podatkov indikatorjev nam ta programski vmesnik omogoča 
tudi dostop do večine metapodatkov, s katerimi lahko presejamo in natančneje
določimo našo poizvedbo in ki vklju"cujejo::
\begin{itemize}
\item viri podatkov in njihovi opisi 
  \angl{Catalog Source Queries~\fnurl{http://api.worldbank.org/sources?format=json}},
\item seznam držav, skupin držav in regij z identifikatorji 
  \angl{Country Queries~\fnurl{http://api.worldbank.org/countries?format=json}},
\item razdelitev višin dohodkov z identifikatorji 
  \angl{Income Level Queries~\fnurl{http://api.worldbank.org/incomeLevels?format=json}},
\item seznam indikatorjev 
  \angl{Indicator Queries~\fnurl{http://api.worldbank.org/indicators?format=json}},
\item seznam tipov posojil 
  \angl{Lending Type Queries~\fnurl{http://api.worldbank.org/lendingTypes?format=json}},
\item seznam tem 
  \angl{Topics~\fnurl{http://api.worldbank.org/topics}}.
\end{itemize}

Za pridobitev podatkov indikatorjev potrebujemo metapodatke o indikatorjih in
državah. Primere teh metapodatkov si bomo podrobneje pogledali v nadaljevanju.

Ker je mogoče z eno poizvedbo dostopati do velike količine podatkov, ima
programski vmesnik za dostop do podatkov indikatorjev implementirano
% "stevil"cenje,
% oštevilčenje,
paginacijo,
% deljenje na strani,
s katero je omejeno število podatkov, ki jih lahko dobimo z eno
poizvedbo. Tako so podatki razdeljeni na skupine, ki jih imenujemo strani.

Vsi odgovori na veljavne poizvedbe po podatkih in metapodatkih, ki so na voljo
s programskim vmesnikom indikatorjev razvoja, imajo enako osnovno obliko. 
Poizvedbe vračajo seznam z dvema elementoma, kjer ima prvi element 
informacije o količini podatkov in trenutnem izboru podatkov, drugi element 
pa vsebuje seznam izbranih podatkov (primer \ref{basic_response}). Privzeta
vrednost števila elementov na stran je $50$, kar lahko spremenimo tako, da
poizvedbi nastavimo parameter GET \verb|per_page| na poljubno vrednost. Če
želimo pridobiti podatke z več strani, moramo za vsako stran poslati novo
poizvedbo, v kateri podamo številko želene strani s parametrom GET \verb|page|.
Veljavne poizvedbe s sitom, ki ne vrača nobenih podatkov, imajo vrednost 
drugega elementa osnovnega seznama \verb|null|.
Za neveljavne poizvedbe pa programski vmesnik vrača seznam z enim elementom,
ki vsebuje podatke o napaki poizvedbe (primer \ref{error_response}).


\begin{snippet}
\begin{center}
\begin{lstlisting}
[
    {
        'page': 1,
        'pages': 137,
        'per_page': '50',
        'total': 6831
    },
    [
        <podatki>,
        ...
    ]
]
\end{lstlisting}
\end{center}
\caption{Osnovna oblika odgovora programskega vmesnika Svetovne banke za
veljavno poizvedbo indikatorjev.} 
\label{basic_response}
\end{snippet} 


\begin{snippet}
\begin{center}
\begin{lstlisting}
[
    {
        'message': [
            {
                'id': '120',
                'key': 'Parameter \'country\' has an invalid value',
                'value': 'The provided parameter value is not valid'
            }
        ]
    }
]
\end{lstlisting}
\end{center}
\caption{Osnovna oblika odgovora programskega vmesnika Svetovne banke za
neveljavne poizvedbe.}
\label{error_response}
\end{snippet} 


\subsection{Opis seznama indikatorjev}

Programski vmesnik Svetovne banke za indikatorje razvoja nam ponuja seznam 
vseh indikatorjev z imeni, opisi, kodami in drugimi metapodatki 
(primer \ref{indicator_response}). Programski vmesnik nam omogoča tudi dostop
do podatkov posameznega indikatorja določenega s kodo in presejanje seznama 
indikatorjev glede na vir podatkov \ref{indicator_queries}. V našem programu
smo uporabili le poizvedbo za celoten seznam indikatorjev, da smo omogočili 
iskanje in presejanje po vseh poljih indikatorjev.


\begin{snippet}
\begin{center}
\begin{lstlisting}
http://api.worldbank.org/indicators?format=json
http://api.worldbank.org/indicators?format=json&source=5
http://api.worldbank.org/indicators/A10i?format=json
\end{lstlisting}
\end{center}
\caption{Primeri poizvedb po seznamu indikatorjev.
1) seznam vseh indikatorjev, 2) seznam indikatorjev glede na vir podatkov,
3) podatki indikatorja ``A10i''}
\label{indicator_queries}
\end{snippet} 


\begin{snippet}
\begin{center}
\begin{lstlisting}
{
    'id': '1.0.HCount.2.5usd',
    'name': 'Poverty Headcount (\$2.50 a day)',
    'source': {
        'id': '37',
        'value': 'LAC Equity Lab'
    },
    'sourceNote': 'The poverty headcount index measures the 
                   proportion of the population with daily per 
                   capita income (in  2005 PPP) below the poverty
                   line.',
    'sourceOrganization': 'LAC Equity Lab tabulations of SEDLAC 
                           (CEDLAS and the World Bank).',
    'topics': [
        {
            'id': '11',
            'value': 'Poverty '
        }
    ]
}
\end{lstlisting}
\end{center}
\caption{Podatki indikatorja 
% TODO
% Ali moram ime indikatorja oznaciti (naprimer velika zacetnica)
% > Podatki indikatorja Stopnja rev...
stopnja revščine pri dohodku 2,5 dolarja na dan.}
\label{indicator_response}
\end{snippet} 


\subsection{Opis seznama držav}

Seznam držav na programskem vmesniku Svetovne banke vsebuje podatke o imenih, 
opisih, kodah dr"zav po standardu ISO-3166-1, regijah in druge metapodatke 
(primer \ref{country_response}). Programski
vmesnik nam omogoča tudi presejanje seznama držav po kodi države, regiji,
višini dohodka in tipu posojil (primer \ref{country_queries})



\begin{snippet}
\begin{center}
\begin{lstlisting}
http://api.worldbank.org/countries?format=json
http://api.worldbank.org/countries/svn?format=json
http://api.worldbank.org/countries?format=json&incomeLevel=HIC&region=ECS
\end{lstlisting}
\end{center}
\caption{Primeri poizvedb po seznamu držav.
1) seznam vseh držav, 2) podatki ene države,
3) seznam držav v Evropi in Osrednji Aziji z visoko višino dohodka.}
\label{country_queries}
\end{snippet} 


Ta seznam ne vsebuje zgolj držav, ampak tudi regije in skupine držav, 
združenih glede na različne kriterije, kot so višina dohodka, velikost, stopnja
razvoja. Poleg tega zgornji seznam vsebuje tudi nekatere izjeme, kot je trenutno
Kosovo. V nadaljevanju bomo za vse naštete tipe lokacijskih podatkov
uporabljali besedo ``države''. 


\begin{snippet}
\begin{center}
\begin{lstlisting}
{
    'id': 'ABW',
    'iso2Code': 'AW',
    'name': 'Aruba',
    'region': {
        'id': 'LCN',
        'value': 'Latin America & Caribbean '
    },
    'adminregion': {
        'id': '',
        'value': ''
    },
    'incomeLevel': {
        'id': 'HIC',
        'value': 'High income'
    },
    'lendingType': {
        'id': 'LNX',
        'value': 'Not classified'
    },
    'capitalCity': 'Oranjestad',
    'longitude': '-70.0167',
    'latitude': '12.5167'
},
\end{lstlisting}
\end{center}
\caption[some]{Izsek podatkov veljavne poizvedbe držav.}
\label{country_response}
\end{snippet} 


\subsection{Dostop do podatkov indikatorjev}

Za dostop do podatkov posameznega indikatorja potrebujemo kodo
indikatorja s seznama vseh indikatorjev in kodo ene ali več držav. Namesto
kode ene ali več držav, lahko uporabimo tudi ključno besedo {\tt all}, ki
označuje vse kode držav. Pri večjih količinah podatkov lahko z dodatnimi
parametri določimo število podatkov na stran in želeno stran podatkov.
Primer \ref{indicator_dataset_request} prikazuje osnovno obliko poizvedbe,
kjer so:
\begin{description}
  \item [\tt country] s podpičjem ločen seznam kod izbranih držav, ki jih 
    preberemo iz polja {\tt id} ali {\tt iso2Code}, ki sta prikazana v primeru 
    \ref{country_response}, ali pa ključna beseda {\tt all},
  \item [\tt indicator\_id] polje {\tt id} indikatorja ki je prikazano v primeru 
    \ref{indicator_response},
\item [\tt parametri] Dodatni parametri GET 
\end{description}
Za poizvedbe do podatkov indikatorjev so poleg osnovnih parametrov GET 
\verb|per_page|, \verb|page| in \verb|format|, opisanih v poglavju 
\ref{sec:podatki_ind_razvoja}, na voljo tudi dodatni parametri za presejanje
rezultatov poizvedbe:
\begin{description}  
\item [\tt mrv] Številska vrednost, ki določi maksimalno število zadnjih meritev,
    ki jih programski vmesnik vrne. Ko uporabljamo polje \verb|mrv|, bo 
    programski vmesnik izpustil ničelne vrednosti za obdobja v katerih ni
    meritev.
\item [\tt gapfill] Zastavica \verb|y| ali \verb|n| za manjkajoče vrednosti meritev.
    Vrednost \verb|y| v kombinaciji s poljem \verb|mrv| poskrbi, da programski 
    vmesnik ne izpusti nobenega časovnega intervala.
\item [\tt date] Polje oblike \verb|'leto'| ali \verb|'leto:leto'| ki omeji rezultate poizvedbe
    na določeno leto ali interval med določenimi leti. 
\end{description}


\begin{snippet}
\begin{center}
\begin{lstlisting}
http://api.worldbank.org/en/countries/<country>/indicators/<indicator_id>?<parametri>
\end{lstlisting}
\end{center}
\caption{Osnovna oblika poizvedbe za podatke enega indikatorja.}
\label{indicator_dataset_request}
\end{snippet} 

% http://api.worldbank.org/countries/svn/indicators/SL.TLF.ACTI.FE.ZS?format=json
% http://api.worldbank.org/countries/svn;usa/indicators/SL.TLF.ACTI.FE.ZS?format=json
% http://api.worldbank.org/countries/all/indicators/SL.TLF.ACTI.FE.ZS?format=json

Privzeta vrednost za količino podatkov na stran \verb|per_page| je $50$. 
Zgornja meja pa ni strogo določena, vendar je odvisna od velikosti odgovora. 
Ugotovili smo, da se zanesljivost programskega vmesnika manjša z večjo 
količino podatkov na stran. V našem programu smo se omejili na $1000$ podatkov
na stran, kar se je izkazalo za uporabno razmerje med hitrostjo in 
zanesljivostjo programskega vmesnika. Privzeto bo programski vmesnik vrnil 
podatke za vse časovne vrednosti. V odgovoru programskega vmesnika dobimo seznam objektov 
(primer \ref{dataset_response}) z datumom, indikatorjem, državo in vrednostjo.

\begin{snippet}
\begin{center}
\begin{lstlisting}
{
    'indicator': {
        'id': 'SP.POP.TOTL',
        'value': 'Population, total'
    },
    'country': {
        'id': 'IL',
        'value': 'Israel'
    },
    'value': '6289000',
    'decimal': '0',
    'date': '2000'
}
\end{lstlisting}
\end{center}
\caption{Podatki za indikator SP.POP.TOTL (skupno število prebivalcev države) za Izrael leta
2000.}
\label{dataset_response}
\end{snippet} 

Slabosti programskega vmesnika indikatorjev Svetovne banke za uporabo v namene
podatkovnega rudarjenja so v tem, da vmesnik ni namenjen prenosu večje 
količine podatkov z eno samo poizvedbo. Zaradi paginacije moramo za en sam 
indikator narediti več poizvedb, da prenesemo podatke z vseh strani. Prav 
tako podatkovni vmesnik ne podpira poizvedb po več indikatorjih hkrati, kar
potrebujemo za iskanje zakonitosti med posameznimi indikatorji.

\section{Podatki podnebnih meritev}

Programski vmesnik Svetovne banke za podnebne podatke omogoča dostop do 
podatkov napovednih modelov in zgodovinskih meritev meteoroloških postaj. V tej 
diplomski nalogi smo se odločili uporabiti samo podatke zgodovinskih meritev, 
saj si s temi podatki lahko uporabnik programa Orange sam sestavi svoje 
napovedne modele.

Za razliko od uporabe programskega vmesnika indikatorjev, lahko pri tem
programskem vmesniku uporabljamo veljavne kode držav ISO 3166-1 alpha-2 ali ISO 3166-1 
alpha-3, ali pa številski identifikator vodotočnega 
območja.

Ta programski vmesnik nam omogoča dostop do podatkov o povprečnih 
temperaturah in padavinah v časovnih obdobjih enega leta, desetletja ali pa 
nam omogoča dostop do mesečnih povprečij skozi vsa leta meritev.


\subsection{Dostop do podatkov podnebnih meritev}

Za dostop do podnebnih podatkov preko programskega vmesnika Svetovne banke
potrebujemo kodo države ISO-3166-1 alpha-3 ali številski identifikator
vodotočnega območja (slika \ref{climate_data_api_basins}). Programski vmesnik
nam omogoča dostop do meritev povprečnih količin padavin in temperatur za 
letno ali desetletno obdobje. Poleg letnega in desetletnega obdobja nam 
programski vmesnik ponuja tudi povprečno količino padavin in temperatur za 
posamezne mesece skozi vsa leta meritev. Obliko poizvedbe prikazuje primer 
\ref{climate_dataset_request}, kjer je:
\begin{description}
\item [\tt loc\_type] vrsta identifikatorja območja (``basin'' za vodotočno območje, 
  ``country'' za države),
\item [\tt data\_type] vrsta meritev (``pr'' za padavine, ``tas'' za temperature),
\item [\tt interval] vrsta meritvenega obdobja (``month'' za mesečno, ``year'' za letno in
  ``decade'' za desetletno),
\item [\tt location] koda države ali številski identifikator vodotočnega območja.
\end{description}
Za razliko od programskega vmesnika indikatorjev, nam programski vmesnik
podnebnih meritev z eno poizvedbo omogoča dostop do podatkov le za eno državo.
To pomeni, da je količina podatkov dovolj omejena, da nam programski vmesnik
vedno vrne vse podatke brez paginacije, kot prikazuje primer 
\ref{climate_dataset_response}.


\begin{figure}
\begin{center}
\includegraphics[width=13.75cm]{pic/climate_data_api_basins.png}
\end{center}
\caption{Prikaz vodotočnih območij sveta.}
\label{climate_data_api_basins}
\end{figure} 


\begin{snippet}
\begin{center}
\begin{lstlisting}
http://climatedataapi.worldbank.org/climateweb/rest/v1/<loc_type>/cru/<data_type>/<interval>/<location>
\end{lstlisting}
\end{center}
\caption{Osnovna oblika poizvedbe za podnebne podatke.}
\label{climate_dataset_request}
\end{snippet} 


\begin{snippet}
\begin{center}
\begin{lstlisting}
[
    {
        'month': 0,
        'data': 68.93643
    },
    {
        'month': 1,
        'data': 64.23069
    },
    {
        'month': 2,
        'data': 81.098724
    },
    ...
]
\end{lstlisting}
\end{center}
\caption{Primer odgovora za poizvedbo količine padavin v posameznih mesecih v 
  Sloveniji.}
\label{climate_dataset_response}
\end{snippet} 






\section{Težave pri uporabi programskih vmesnikov Svetovne banke}
\label{api_gotchas}


Programski vmesniki Svetovne banke zajemajo podatke iz različnih virov, zato je
težko zagotoviti pravilnost in konsistentnost podatkov. Poleg tega pa se 
programski vmesnik in spletna stran z dokumentacijo občasno spremenita, kar
povzroča še dodatne težave pri uporabi. Nekatere težave, ki smo jih opazili
so:
\begin{itemize}  
\item nekaterim delom dokumentacije se je med izdelavo te diplomske naloge
  spremenil spletni naslov, tako da do tistih delov sedaj nimamo več dostopa,
\item polje za datum \verb|date| v odgovoru je opisano, vendar niso dokumentirane vse možne vrednosti
    (nekaj primerov nedokumentiranih vrednosti:
    ``last known value'' ``2001 - 2015'' ``2040''),
\item delovanje sita z različnimi kombinacijami polj \verb|mrv|, \verb|gapfill|
  in \verb|date| ni ustrezno opisano,
\item v odgovoru poizvedbe po podatkih indikatorjev ponekod manjkajo vrednosti
  kot so koda države, ime države ali ime indikatorja,
\item zgornja meja števila izbranih lokacij na $250$ ni navedena, prav tako pa ni
  dokumentirana napaka, ki jo v tem primeru vrne programski vmesnik,
\item nemogoče je ugotoviti pogostost vzorčenja indikatorja \verb|frequency|.
\end{itemize}  









\chapter{Knjižnica in gradniki za Orange}

V okviru diplomske naloge smo razvili tri ločene komponente za programerje in
končne uporabnike programa Orange. Prva komponenta je
dostopna kot samostojni paket 
\verb|simple_wbd|\fnurl{https://pypi.python.org/pypi/simple_wbd/0.5.1}. 
Druga in tretja komponenta pa sta združeni v paketu
\verb|orange3-datasets|\fnurl{https://pypi.python.org/pypi/Orange3-Datasets/0.1.3}.


V nalogi razvita programska knjižnica \verb|simple_wbd|
omogoča enostaven dostop do programskega vmesnika indikatorjev in podnebnih
podatkov Svetovne banke. Knjižnico smo implementirali z uporabo čim manjšega
števila odvisnosti in je 
% TODO ali je python z veliko
namenjena splošni uporabi v programih napisanih v jeziku Python. Poudarka pri zasnovi knjižnice 
\verb|simple_wbd| sta predvsem enostavnost razširitve in zanesljivost. Ta cilja
dosežemo z mehanizmom za vključevanje lastne kode v komponente knjižnice
in mehanizmi za popravljanje ali odstranjevanje okvarjenih podatkov.


Drugi sestavni del je razširitev knjižnice \verb|simple_wbd| s 
funkcionalnostmi potrebnimi za lažje delo v programu Orange. To predvsem 
zavzema pretvorbo pridobljenih podatkov v podatkovno tabelo Orange in tabelo 
numpy. Ta sklop je namenjen skriptnemu delu s programom Orange 
\cite{orange_scripting} in je dostopen
kot modul \verb|api_wrapper| v programskem jeziku Python znotraj paketa \verb|orangecontrib.wbd|.


Tretji sestavni del na"se re"sitve je grafični vmesnik za uporabo \verb|api_wrapper| modula.
Namen grafičnega vmesnika je omogočiti dostop do podatkov 
programskega vmesnika Svetovne banke znotraj programa Orange za namen obdelave,
analize in iskanja zakonitosti v podatkih s pomo"cjo vizualizacij in ostalih grafi"cnih gradnikov,
ki jih Orange ponuja.

\section{Knjižnica simple\_wbd}

Knjižnica \verb|simple_wbd| programerjem olajša dostop do podatkov 
programskega vmesnika Svetovne banke. Glavni namen knjižnice je 
združevanje večjega števila zahtev po podatkih in enostavna predstavitev 
prejetih rezultatov. Te rezultate je nato iz več dimenzij možno pretvoriti v 
dvo-dimenzionalno polje, primerno za uporabo v programu Orange. Glavna 
razreda te knjižnice sta \verb|IndicatorAPI| in \verb|ClimateAPI|. Prvi 
omogoča pridobivanje podatkov iz programskega vmesnika indikatorjev, drugi pa 
s programskega vmesnika podnebnih meritev.


Čeprav za dostop do programskega vmesnika Svetovne banke že obstajajo 
rešitve, kot sta knjižnici 
\verb|wbdata|\fnurl{https://pypi.python.org/pypi/wbdata/0.2.7} in
\verb|wbpy|\fnurl{https://pypi.python.org/pypi/wbpy/2.0.1}, smo se odločili
za lastno implementacijo podobne knjižnice. Glavni razlog za to je, da
obstoječe rešitve poskušajo čim bolj natančno predstaviti programski
vmesnik Svetovne banke, ne pa olajšati dostop do čim večje količine
podatkov.

Za potrebe te knjižnice smo razvili lastno rešitev za predpomnjenje poizvedb,
saj so se bolj splošne rešitve, kot na primer
\verb|vcrpy|\fnurl{https://pypi.python.org/pypi/vcrpy/1.10.0} in
\verb|requests-cache|\fnurl{https://pypi.python.org/pypi/requests-cache/0.4.12},
izkazale za prepočasne, ko delamo z večjimi količinami podatkov. Naša
rešitev za predpomnjenje izkorišča dejstvo, da je vsaka poizvedba določena le
z naslovom URL in da so vsi odgovori oblike JSON. Za vsak URL naredimo novo
datoteko v sistemskem začasnem imeniku, v kateri hranimo serializirane JSON
podatke. Ker se podatki na programskem vmesniku Svetovne banke redko
posodabljajo, smo za čas veljavnosti začasnih datotek izbrali en teden.

% NOTE: serailizacija je baje uredu beseda:
% http://eprints.fri.uni-lj.si/2711/1/63100205-VALENTIN_KRAGELJ-\
% Pregled_in_analiza_tehnologij_za_serializacijo_objektov.pdf


\subsection{Razred IndicatorAPI}
\label{razered_indicator_api}

\verb|IndicatorAPI| je razred namenjen pridobivanju podatkov indikatorjev
razvoja držav. Ker ima programski vmesnik Svetovne banke omejitev koliko 
podatkov lahko prenesemo z eno poizvedbo in nam dovoli tvoriti poizvedbe le za
en indikator hkrati smo napisali razred, ki v ozadju tvori in izvede
poizvedbe za vse strani vseh zahtevanih indikatorjev. Po
prvi poizvedbi za en indikator se na"sa re"sitev sprehodi čez število preostalih strani 
% (primer \ref{basic_response})
, ki so na voljo, in pridobljene podatke več
strani združi in predstavi kot rezultat ene same poizvedbe. Ta postopek ponovi
za vse zahtevane indikatorje in njihove rezultate vrne v obliki slovarja, ki 
ima za ključ kodo indikatorja posamezne zahteve.
Poleg tega, da skrbi za prenos vseh strani podatkov, tudi beleži število 
izvedenih in število potrebnih poizvedb za celoten prenos. Ta števila se
lahko uporablja za prikaz napredka prenosa podatkov.


Za namene uporabe v razredu \verb|IndicatorAPI| smo v knjižnici \verb|simple_wbd|
razvili mehanizme za odpravo nekaterih napak omenjenih v poglavju 
\ref{api_gotchas}.

Pri manjkajočih vrednostih držav v poizvedbah za podatke indikatorjev
poskušamo določiti pravilne vrednosti. To naredimo s pomočjo dveh
slovarjev: prvi slika kode držav v imena, drugi pa imena držav v kode. V
primeru manjkajoče vrednosti kode ali imena, poskušamo to prebrati iz enega
od naštetih slovarjev. Če nam ne uspe ugotoviti manjkajočih vrednosti,
trenutni vnos odstranimo iz rezultata poizvedbe.

Drugi tip napak, ki ga lahko delno popravimo, so napačne vrednosti v polju
\verb|date| v poizvedbah za podatke indikatorjev. Ker lahko v temu polju
pričakujemo poljubno besedilo, dela naš pretvornik za polje \verb|date| v 
datum, tako da poskuša v datum pretvoriti čim daljšo predpono besedila. Naprimer, 
v besedilu ``2005Q1 - 2006Q2'' je najdalj"sa predpona, ki "se ozna"cuje veljaven datum opisan
v dokumentaciji polja {\tt date}, 
predpona ``2005Q1``.
Če nam ne uspe besedila pretvoriti v veljaven datum, trenutni vnos odstranimo
iz rezultata poizvedbe.


\ \\
Glavne metode ki jih ponuja razred IndicatorAPI so:

\begin{description}  
\item [\tt get\_indicators] za pridobivanje seznama indikatorjev s kodami, imeni
      in opisi,
\item [\tt get\_countries] za pridobivanje seznama držav z metapodatki,
\item [\tt get\_dataset] za pridobivanje instance razreda \verb|IndicatorDataset|,
      ki vsebuje podatke indikatorjev.
\end{description}

Ena izmed lastnosti razreda \verb|IndicatorAPI| je, da mu lahko ob
inicializaciji podamo razred v katerem želimo prejeti rezultat poizvedbe. Ta
razred mora dedovati od osnovnega razreda \verb|IndicatorDataset|. Na ta
način lahko enostavno razširimo funkcionalnost \verb|simple_wbd| knjižnice.
V primeru \ref{indicator_api_extend} vidimo en način za razširitev razreda 
\verb|IndicatorDataset|, tako da uporabniku razreda \verb|MyIndicatorAPI| ni
potrebno izrecno podati razreda {\tt MyIndicatorDataset} v konstruktor.

\begin{snippet}
\begin{center}
\begin{lstlisting}
class MyIndicatorDataset(simple_wbd.IndicatorDataset):
    
    def as_numpy(self):
        raise NotImplemented()
    
    def as_orange_table(self):
        raise NotImplemented()

class MyIndicatorAPI(simple_wbd.IndicatorAPI):

    def __init__(self):
        super().__init__(MyIndicatorDataset)
\end{lstlisting}
\end{center}
\caption[some]{Primer razširitve osnovnega razreda rezultatov poizvedb.}
\label{indicator_api_extend}
\end{snippet} 


\subsubsection{Razred IndicatorDataset}
\label{razred_indicatordatasets}

Razred \verb|IndicatorDataset| je osnovni razred v katerem dobimo zahtevane 
podatke indikatorjev. Ta razred vsebuje vse potrebne metode in podatke za 
predstavitev rezultatov programskega vmesnika na dva načina: kot slovar
rezultatov poizvedb za posamezen indikator in dvo dimenzionalen seznam. 
Posamezna vrednost v teh podatkih je določena z državo, časovno komponento in 
kodo indikatorja. 

Podatke lahko predstavimo kot dvodimenzionalno polje v dveh oblikah: kot
časovne vrste ali kot podatki držav. Obliko predstavitve izberemo s
parametrom \verb|time_series| metode \verb|as_list|. Za predstavitev obeh oblik
je prva vrstica polja uporabljena kot naslovna vrstica, ki opisuje podatke v 
stolpcih.

Ko uporabljamo obliko časovnih vrst, so elementi prve vrstice kartezični
produkt kod indikatorjev in držav. V prvem stolpcu polja pa imamo časovno
komponento podatkov. Na ta način so vsi ostali elementi polja določeni s 
časovno komponento, državo in kodo indikatorja.

Ko dostopamo do dvodimezionalnega polja, ki predstavlja podatke držav, pa je v
prvi vrstici kartezični produkt kod indikatorjev in časovne komponente. Prvi
stolpec v tej predstavitvi vsebuje imena držav. Za razliko od predstavitve v 
obliki časovnih vrst, v to polje vstavimo "se dodatne stolpce, ki vsebujejo
metapodatke dr"zav iz primera \ref{country_response}: regija \verb|region|, 
administrativna regija \verb|adminregion|, višina dohodka \verb|incomeLevel|, 
vrsta posojil \verb|lendingType|, geografska širina \verb|latitude|, 
geografska dolžina \verb|longitude|. Tudi tukaj so vsi ostali elementi določeni s
časovno komponento, državo in kodo indikatorja. 


\subsection{Razred ClimateAPI}

Razred \verb|ClimateAPI| olajša dostop do podnebnih podatkov programskega
vmesnika Svetovne banke. Ta programski vmesnik dovoli poizvedbe po podatkih le 
ene vrste meritev za eno vrsto meritvenega obdobja in eno državo. Naš razred 
naredi kartezični produkt med vsemi zahtevanimi vrstami meritev, vrstami
meritvenih obdobij in državami. Nato iz tega zgradi in izvede vse poizvedbe 
in predstavi podatke kot enotni odgovor. V razredu \verb|ClimateAPI| hranimo 
tudi število vseh potrebnih poizvedb in število že izvedenih poizvedb, kar 
lahko uporabimo za prikaz napredka prenosa podatkov.



\subsubsection{Razred ClimateDataset}

Razred \verb|ClimateDataset| je osnovni razred v katerem dobimo zahtevane 
podatke podnebnih meritev. Vsebuje vse potrebne metode in podatke za 
predstavitev rezultatov programskega vmesnika na dva glavna načina: kot
gnezden slovar in dvodimenzionalen seznam. Posamezna vrednost v teh podatkih
je določena z državo, vrsto podatkov in časovno komponento. Poleg omenjenih
načinov predstavitve podatkov lahko dostopamo tudi do neobdelanih podatkov 
prejetih iz programskega vmesnika za vsako poizvedbo posebej.

Časovno komponento rezultata sestavljata vrsta meritvenega obdobja in 
začetek obdobja meritve. Sestavljeno časovno komponento uporabljamo, da se 
izognemo dvoumnim primerom vrednosti začetka obdobja za letni in desetletni 
interval meritev. Primera takih dveh časovnih obdobij sta 
\verb|'decade - 1990'| in \verb|'year - 1990'|.


Do podatkov predstavljenih z gnezdenim slovarjem lahko dostopamo preko funkcije
\verb|as_dict|. V tej funkciji združimo podatke poizvedb programskega
vmesnika v gnezden slovar s štirimi nivoji gnezdenja: država, vrsta meritev,
vrsta meritvenega obdobja in obdobje meritve. Zadnji nivo gnezdenja vsebuje
vrednosti podnebnih meritev.

Pri predstavitvi podatkov kot dvodimenzionalno polje moramo dve od treh
komponent podatkov (država \verb|'country'|, vrsta podatkov \verb|'type'|, 
in časovna komponenta \verb|'interval'|)
združiti in ju skupaj prikazati v vrsticah ali stolpcih. Za razliko od razreda
\verb|IndicatorDataset|, ki podpira le dve obliki prikaza, lahko v razredu
\verb|ClimateDataset| sami določimo katere komponente bodo v stolpcih in
katere v vrsticah. Primer \ref{list_configurations} prikazuje različne mo"znosti izborov komponent.
Spremenljivki \verb|list1| ind \verb|list2| iz
prejšnjega primera prikazujeta privzeto konfiguracijo, kjer imamo v stolpcih
kartezični produkt vrst meritev in vrst meritvenih obdobij, v vrsticah pa
podatke države. Spremenljivka \verb|list4| prikazuje konfiguracijo za
predstavitev v obliki časovnih vrst.

\begin{snippet}
\begin{center}
\begin{lstlisting}
import simple_wbd

api = simple_wbd.ClimateAPI()                   
climate_dataset = api.get_instrumental(['svn', 'usa', 'aus'])

list1 = ds.as_list()
list2 = ds.as_list(columns=['type', 'interval'])  # default  value
list3 = ds.as_list(columns=['type'])
list4 = ds.as_list(columns=['type', 'country']) 
list5 = ds.as_list(columns=['country'])
\end{lstlisting}
\end{center}
\cprotect
\caption{Prikaz nekaj možnih oblik dvodimezionalnega polja vrednosti.} 
\label{list_configurations}
\end{snippet} 



\section{Modul api\_wrapper}


Znotraj paketa \verb|orangecontrib.wbd| smo razvili modul \verb|api_wrapper| v
katerem smo razširili razreda \verb|IndicatorDataset| in \verb|ClimateDataset|
na način, ki je prikazan v primeru \ref{indicator_api_extend}. Naša
razširitev obema razredoma doda metodi za pretvorbo podatkov v podatkovno 
tabelo Orange in tabelo numpy.

\subsection{Razširitev razreda IndicatorDataset}

Glavne funkcionalnosti, za uporabo programskega vmesnika indikatorjev, so
vključene v naši razširitvi razreda \verb|IndicatorDataset|. To je na prvem
mestu metoda \verb|as_numpy_array|, ki rezultat metode \verb|as_list|
opisane v poglavju \ref{razred_indicatordatasets}, spremeni v polje numpy in odstrani vse
stolpce, ki ne vsebujejo niti ene veljavne vrednosti. Druga metoda pa je
\verb|to_orange_table|, ki podatke dobljene iz metode \verb|as_numpy_array|,
pretvori v podatkovno tabelo Orange. To tabelo lahko oblikuje kot časovno
vrsto ali pa kot seznam držav. %, kot je opisano v poglavju \ref{razred_indicatordatasets}.
Obliko tabele Orange, ki jo želimo izbrati,
določimo s parametrom \verb|time_series|. % TODO: (angl.\ Orange data table).
Ta metoda tudi poskrbi za pravilno nastavljeno domeno\footnote{Domena
``Domain'' je razred v orodju Orange, ki določa tipe in imena značilk in
ciljnih razredov.} podatkov.


\subsection{Razširitev razreda ClimateDataset}
\label{razsiritev_razreda_climatedataset}

Prav tako kot razširitev razreda \verb|IndicatorDatasets|, tudi ta razširitev
doda metodi \verb|as_numpy_array| in \verb|to_orange_table|. Prav tako kot v
razširitvi razreda \verb|IndicatorDatasets|, lahko tudi tukaj s parametrom
\verb|time_series| izberemo obliko tabele Orange. Pri vrednosti parametra 
\verb|time_series = False| se nastavi privzeta oblika tabele, prikazana kot
\verb|list1|, sicer pa kot \verb|list3|, iz primera \ref{list_configurations}.
S tem parametrom pa izgubimo možnost poljubne oblike tabele Orange.


\section{Grafični vmesnik}

Programski raz"siritvi \verb|Orange3-DataSets| za 
grafični vmesnik programa Orange smo sedaj dodali novo skupino gradnikov imenovano
``Data Sets'' (slika \ref{data_sets_group}). V okviru te naloge smo za skupino
``Data Sets'' izdelali dva ločena gradnika. Prvi gradnik se imenuje ``WB
Climate'' (slika \ref{co2_temp_climate}) in nam preko grafičnega vmesnika 
omogoča dostop do podnebnih podatkov Svetovne banke, drugi gradnik pa se 
imenuje ``WB Indicators'' (slika \ref{co2_temp_indicator}) in nam
preko grafičnega vmesnika omogoča dostop do podatkov indikatorjev razvoja.

Oba grafična vmesnika sta narejena skladno z vodili grafičnih vmesnikov
programa Orange. To smo dosegli tako, da smo za večino elementov grafičnega
vmesnika uporabili predpripravljene gradnike v paketu \verb|Orange.gui|. Pri
gradnji vmesnikov smo bili pozorni na odzivnost grafičnega vmesnika.
Počasne operacije branja podatkov z interneta smo zato prestavili v ločeno nit.
 
\begin{figure}
  \begin{center}
    \includegraphics[width=4cm]{pic/data_sets_group.png}
  \end{center}
  \caption{Skupina gradnikov Data Sets, ki smo jih razvili v pri"cujo"ci nalogi.}
  \label{data_sets_group}
\end{figure} 


\subsection{Gradnik WB Indicators}

\verb|WB Indicators| je gradnik programa Orange za dostop do podatkov
programskega vmesnika indikatorjev. Omogoča nam enostavno izbiro
enega ali več indikatorjev in ene ali več držav, za katere želimo dobiti
podatke izbranih indikatorjev. Za lažje iskanje indikatorjev smo v grafičnem
vmesniku dodali dve možnosti presejanja seznama indikatorjev. Pri prvem situ
si lahko izberemo prikaz vseh indikatorjev, pogosto uporabljenih 
indikatorjev\footnote{Seznam je na voljo na strani
\url{http://data.worldbank.org/indicator?tab=all}}
ali pa izpostavljenih indikatorjev\footnote{Seznam je na voljo na strani
\url{http://data.worldbank.org/indicator?tab=featured}}. Drugo sito pa je
tekstovno presejanje po poljih: koda {\tt id}, ime {\tt name}, teme {\tt topics} in
viri {\tt sources}. 
% TODO: preveri ta stavek
V grafičnem vmesniku si lahko tudi izberemo eno izmed oblik izhodnih podatkov
kot časovne vrste \angl{Time Series} ali podatke držav \angl{Countries}, 
kot smo ju opisali v razdelku \ref{razred_indicatordatasets}. 
Implementirali pa smo tudi prikaz napredka prenosa podatkov s števili vseh in
že izvedenih poizvedb, omenjenih v razdelku \ref{razered_indicator_api}.


\begin{figure}
\begin{center}
\includegraphics[width=13.75cm]{pic/co2_temp_indicator_selection.png}
\end{center}
\caption{Gradnik WB Indicators.}
\label{co2_temp_indicator}
\end{figure} 




\subsection{Gradnik WB Climate}

Gradnik za izbiro podnebnih podatkov podatkovnega vmesnika Svetovne banke nam
ponuja možnosti izbire držav, vrste podatkov in vrste meritvenega obdobja.
Prav tako kot v gradniku WB Indicators, lahko tudi tukaj izberemo obliko izhodnih
podatkov. Možni izbiri oblike izhodnih podatkov sta časovne vrste in podatki
držav, kot smo opisali v poglavju \ref{razsiritev_razreda_climatedataset}. 
Kot dodatno možnost pa imamo v tem grafičnem vmesniku tudi zastavico, ki 
določa ali bomo za države izpisovali imena ali pa kode. Tudi temu grafičnemu
vmesniku smo dodali prikaz napredka prenosa podatkov.



\begin{figure}
\begin{center}
\includegraphics[width=13.75cm]{pic/co2_temp_climate_selection.png}
\end{center}
\caption{Gradnik WB Climate.}
\label{co2_temp_climate}
\end{figure} 

\chapter{Primeri uporabe}


\section{Uporaba modula api\_wrapper}

Enostavno uporabo modula \verb|api_wrapper| s skriptnim delom programa Orange
prikazuje primer \ref{scripting_example}. V temu primeru pogledamo, kako
učinkovito lahko napovemo smrtnost otrok iz raznih indikatorjev zdravja,
okolja in infrastrukture. V vrsticah $5$ do $15$ naredimo poizvedbe po
potrebnih podatkih s programskega vmesnika Svetovne banke. Nato v vrsticah $18$
do $27$ odstranimo vrstice iz tabele, ki nimajo ciljne vrednosti in naredimo novo tabelo z
razredom, ki ga želimo napovedovati. Vrednosti, ki jih želimo napovedovati, se
nahajajo v stolpcu $55$ v tabeli \verb|class_data|. Ta stolpec vsebuje podatke
o smrtnosti otrok mlajših od enega leta za leto 2015. V naslednjih vrsticah
pa zgradimo tri napovedne modele: naključni gozd z
regresijskimi drevesi \verb|rf|, linearna regresija z regularizacijo 
\verb|ridge| in srednja vrednost \verb|mean|.
Za ocene napovednih modelov smo uporabili oceni
$RMSE$\fnurl{https://en.wikipedia.org/wiki/Root-mean-square\_deviation} in 
$R^2$~\fnurl{https://en.wikipedia.org/wiki/Coefficient\_of\_determination}.
Rezultate primera \ref{scripting_example} lahko vidimo v tabeli 
\ref{rezultati_skripte}.


\begin{snippet}
\begin{center}
\lstinputlisting{example.py}
\end{center}
\cprotect
\caption{Napovedovanje smrtnosti otrok do enega leta iz podatkov o dostopnosti
  čiste vode, številu bolniških postelj na 1000 prebivalcev in odstotku
  cepljenih otrok do drugega leta starosti.}
\label{scripting_example}
\end{snippet} 

\begin{table}
\begin{center}

\begin{tabular}{l|r|r}
  Learner & RMSE & R2 \\ \hline
  rf & 9.74 & 0.79 \\
  ridge & 17.76 & 0.31 \\
  mean & 21.35 & -0.00
\end{tabular}
\end{center}
\cprotect
\caption{Rezultati napovedi smrtnosti otrok do enega leta starosti.}
\label{rezultati_skripte}
\end{table} 



\section{Napoved temperature s pomočjo $CO_2$ izpustov v ZDA}


Podatke svetovne banke lahko uporabimo tudi kot časovne vrste z uporabo
posebnih gradnikov za delo s časovnimi vrstami \cite{time_series}. Tukaj si
bomo ogledali enostaven primer napovedi temperature v ZDA s pomočjo podatkov o
izpustih $CO_2$. V tej napovedi smo uporabili podatke tako z gradnika 
WB Indicators (Slika \ref{var_indicator_select})
kot tudi z gradnika WB Climate (Slika \ref{var_climate_select}). Podatke obeh
gradnikov smo zdru"zili z gradnikom ``Merge Data'' po obeh "casovnih
komponentah. Nato smo odstranili vnose "casovnih obdobij za katere nimamo na
voljo vseh podatkov. Sestavljeno tabelo prikazuje slika \ref{var_data_table}.
Iz teh podatkov nato zgradimo "casovno vrsto in s pomočjo modela vektorske 
autoregresije VAR \cite{var_model} napovemo podatke za povprečno 
letno temperaturo za naslednjih nekaj let, kar je prikazano na sliki 
\ref{var_forecast_graph}.

\begin{figure}
\begin{center}
\includegraphics[width=13.75cm]{pic/var_setup.png}
\end{center}
\caption{Prikaz povezave gradnikov za napoved temperature.}
\label{var_setup}
\end{figure} 


\begin{figure}
\begin{center}
\includegraphics[width=13.75cm]{pic/var_indicator_select.png}
\end{center}
\caption{Izbor indikatorja $CO_2$ izpustov v ZDA.}
\label{var_indicator_select}
\end{figure} 

\begin{figure}
\begin{center}
\includegraphics[width=8cm]{pic/var_climate_select.png}
\end{center}
\caption{Izbor podatkov povprečnih letnih temperatur v ZDA.}
\label{var_climate_select}
\end{figure} 

\begin{figure}
\begin{center}
\includegraphics[width=10cm]{pic/var_data_table.png}
\end{center}
\caption{Podatkovna tabela s ciljnim razredom, in dvema poljema.}
\label{var_data_table}
\end{figure} 

\begin{figure}
\begin{center}
\includegraphics[width=13.75cm]{pic/var_forecast_graph.png}
\end{center}
\caption{Prikaz napovedi gibanja povprečnih letnih temperatur ``USA - tas'' in
  $CO_2$ izpustov ``United States''.}
\label{var_forecast_graph}
\end{figure} 




\section{Gru"cenje dr"zav}


Podatke, ki jih dobimo z na"sim dodatkom, lahko v programu Orange uporabimo tudi
za grafi"cni prikaz statistik in povezav med dr"zavami. Kot mo"zen primer
uporabe (Slika \ref{clustering_setup}) smo prikazali gru"cenje dr"zav svetovnih regij glede na naslednje
indikatorje (Slika \ref{clustering_indicator_selection}):
\begin{itemize}
  \item odstotek ljudi ki "zivijo v urbanem okolju 
    \angl{Urban population (\% of total)},
  \item smrtnost na $1000$ "zivorojenih otrok
    \angl{Mortality rate, infant (per 1,000 live births)},
  \item "stevilo bolni"skih postelj na $1000$ prebivalcev
    \angl{Hospital beds (per 1,000 people)},
  \item dele"z BDP izdatkov za raziskave in razvoj
    \angl{Research and development expenditure (\% of GDP)},
  \item "stevilo prebivalstva pod pragom rev"s"cine pri meji $3.10$ dolarjev na dan
    \angl{Poverty gap at $\$3.10$ a day (2011 PPP) (\%)}.
\end{itemize}
Med temi podatki teh indikatorjev (slika \ref{clustering_data}) smo izra"cunali evklidsko razdaljo in za prikaz 
uporabili "ze obstoje"ca gradnika programa Orange
``MDS'' (slika \ref{clustering_mds}) in
``Hierarchical Clustering'' (slika \ref{clustering_hierarchial_countries}).


% ID: GB.XPD.RSDV.GD.ZS 
% ID: SH.MED.BEDS.ZS 
% ID: SI.POV.GAP2 
% ID: SP.DYN.IMRT.IN 
% ID: SP.URB.TOTL.IN.ZS 


\begin{figure}
\begin{center}
\includegraphics[width=5cm]{pic/clustering_setup.png}
\end{center}
\caption{Postavitev okolja za prikaz gru"cenja.}
\label{clustering_setup}
\end{figure} 

\begin{figure}
\begin{center}
\includegraphics[width=13.75cm]{pic/clustering_indicator_selection.png}
\end{center}
\caption{Izbor indikatorjev za gru"cenje.}
\label{clustering_indicator_selection}
\end{figure} 

\begin{figure}
\begin{center}
\includegraphics[width=13.75cm]{pic/clustering_data.png}
\end{center}
\caption{Podatki izbranih indikatorjev.}
\label{clustering_data}
\end{figure} 

\begin{figure}
\begin{center}
\includegraphics[width=13.75cm]{pic/clustering_hierarchial_countries.png}
\end{center}
\caption{Prikaz hierarhi"cnega gru"cenja dr"zav.}
\label{clustering_hierarchial_countries}
\end{figure} 

\begin{figure}
\begin{center}
\includegraphics[width=13.75cm]{pic/clustering_mds.png}
\end{center}
\caption{Prikaz gru"cenja MDS.}
\label{clustering_mds}
\end{figure} 


\chapter{Sklepne ugotovitve}


% zaključek (odstavki: kaj smo naredili in kje je koda, kaj nam to omogoča, 
% kaj bi lahko še naredili)

Z izdelavo dodatka za program Orange smo zaklju"cili delo na diplomski nalogi.
Vsa koda se nahaja na prosto dostopnem repozitoriju GIT na naslovih
\url{https://github.com/zidarsk8/simple_wbd} in
\url{https://github.com/zidarsk8/orange3-data-sets}.
Dodatek je "ze dostopen uporabnikom sistema Orange in ga lahko namestimo s
standardnim vmesnikom za delo z dodatki (slika \ref{addon_install}).



\begin{figure}
\begin{center}
\includegraphics[width=12cm]{pic/addon_install.png}
\end{center}
\caption{Standardni vmesnik za delo z dodatki sistema Orange, ki že prikaže 
dodatek, ki je bil razvit v okviru pričujoče naloge.}
\label{addon_install}
\end{figure} 




Z razvitim dodatkom smo omogo"cili dostop do podatkov programskega vmesnika Svetovne 
banke tako v grafičnem vmesniku kot v skriptnem delu programa Orange. Poleg tega
smo z našim vmesnikom tudi poenotili način dostopa do podatkov Svetovne banke
v programu Orange in s tem olajšali vzdrževanje in posodabljanje kode v
primeru spremembe programskega vmesnika Svetovne banke.



Na"s grafični dodatek za dostop do podatkov indikatorjev lahko nadgradimo tako,
da uporabnikom grafičnega vmesnika omogočimo večjo izbiro oblik izhodnih
podatkov in natančnejše presejanje rezultatov. Dodamo lahko tudi več
metapodatkov na posamezne stolpce tabele Orange, ki nam omogočijo boljšo
predstavnost v ostalih gradnikih Orange. V grafični vmesnik za dostop do
podnebnih podatkov lahko dodamo še možnost izbire vodotočnih območij meritev.
Za boljšo predstavo bi lahko postopek izbire držav, regij in vodotočnih
območij (slika \ref{climate_data_api_basins}) omogočili preko interaktivnega 
zemljevida sveta.

% - dodamo metapodatke tudi climate gradniku
% - boljsa pokritost testov
% 
% 
% - V data sets skupino bi lahko dodali se gradnik za katerega od drugih v uvodu
% nastetih spletnih programskih vmesnikov

\begin{thebibliography}{1}

% Indicator data sources

\bibitem{world_dev_ind} World Development Indicators, The World Bank, (August 2016)
\\ URL: \url{http://data.worldbank.org/data-catalog/world-development-indicators}

\bibitem{glob_fin_dev} Data source: Global Financial Development Database (GFDD), The World Bank. Methodology citation: Martin Čihák, Aslı Demirgüç-Kunt, Erik Feyen, and Ross Levine, 2012. ``Benchmarking Financial Systems Around the World.'' World Bank Policy Research Working Paper 6175, World Bank, Washington, D.C. (Junij 2016)
\\ \url{http://data.worldbank.org/data-catalog/global-financial-development}

% \bibitem{int_debt_stat} International Debt Statistics, The World Bank (December 2015) 
% \\ \url{http://data.worldbank.org/data-catalog/international-debt-statistics}

\bibitem{africa_dev_ind} Africa Development Indicators, The World Bank (Februar 2013)
\\ \url{http://data.worldbank.org/data-catalog/africa-development-indicators}

\bibitem{doing_buseness} Doing Business, The World Bank (http://www.doingbusiness.org) (Julij 2016)
\\ \url{http://data.worldbank.org/data-catalog/doing-business-database}

\bibitem{ent_surveys} Enterprise Surveys, The World Bank (Julij 2016)
\\ \url{http://data.worldbank.org/data-catalog/enterprise-surveys}

\bibitem{mil_dev_goals} Millennium Development Goals, The World Bank (Julij 2016)
\\ \url{http://data.worldbank.org/data-catalog/millennium-development-indicators}

\bibitem{edu_stat} World Bank EdStats (Junij 2016)
\\ \url{http://data.worldbank.org/data-catalog/ed-stats}

\bibitem{gen_stat} Gender Statistics, The World Bank (Julij 2016)
\\ \url{http://data.worldbank.org/data-catalog/gender-statistics}

\bibitem{health_pop_stat} HealthStats, World Bank Group (Julij 2016)
\\ \url{http://data.worldbank.org/data-catalog/health-nutrition-and-population-statistics}

\bibitem{ida_res_mes_sys} IDA Results Measurement System, the World Bank (Julij 2016)
\\ \url{http://data.worldbank.org/data-catalog/IDA-results-measurement}

\bibitem{climate_data} Climatic Research Unit, University of East Anglia
\\ \url{http://www.cru.uea.ac.uk/data}

\bibitem{orange_scripting} Janez Demšar and Tomaž Curk and Aleš Erjavec and Črt Gorup and Tomaž Hočevar and Mitar Milutinovič and Martin Možina and Matija Polajnar and Marko Toplak and Anže Starič and Miha Štajdohar and Lan Umek and Lan Žagar and Jure Žbontar and Marinka Žitnik and Blaž Zupan, ``Orange: Data Mining Toolbox in Python,'' Journal of Machine Learning Research, vol. 14, pp. 2349-2353, 2013.


\bibitem{time_series} Jernej Kernc, ``Orodje za interaktivno analizo časovnih vrst,'' UL FRI, diplomska naloga, 2016


\bibitem{var_model}  Eric Zivot, Jiahui Wang, ``Vector Autoregressive Models for Multivariate Time Series'' Modeling Financial Time Series with S-PLUS, pp. 385-429, 2006.






\bibitem{jezicno} Jure Dimec (2002), Medjezično iskanje dokumentov 
\\ \url{http://clir.craynaud.com/clir/MEDJEZICNOISKANJEDOKUMENTOV.pdf}


% \bibitem{opengl} (Avgust, 2013) OpenGL Overview
% \\ \url{http://www.opengl.org/about/}

\end{thebibliography}


%\begin{thebibliography}{99}
%\bibitem{lf} L.\ Fortnow, ``Viewpoint: Time for computer science to grow up'',
%{\it Communications of the ACM}, št.\ 52, zv.\ 8, str.\ 33--35, 2009.
%\bibitem{dk1} D.\ E.\ Knuth, P. Bendix. ``Simple word problems in universal algebras'', v zborniku: Computational Problems in Abstract Algebra (ur. J. Leech), 1970, str. 263--297.
%\bibitem{lat} L.\ Lamport. {\it LaTEX: A Document Preparation System}. Addison-Wesley, 1986.
%\bibitem{bib} O.\ Patashnik (1998) \BibTeX{}ing. 
%\\ http://ftp.univie.ac.at/packages/tex/biblio/bibtex/contrib/doc/btxdoc.pdf
%\bibitem{licence} licence-cc.pdf. Dostopno na: 
%\end{thebibliography}


\end{document}

