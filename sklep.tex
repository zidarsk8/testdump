\chapter{Sklepne ugotovitve}


% zaključek (odstavki: kaj smo naredili in kje je koda, kaj nam to omogoča, 
% kaj bi lahko še naredili)

Z izdelavo dodatka za program Orange smo zakljucili delo na diplomski nalogi.
Koda izdelanega dodatka se nahaja na git ....


% S tem dodatkom smo omogocili dostop do podatkov Svetovne banke tudi
% uporabnikom programa Orange brez znanja o samem programskem vmesniku SB.
% 
% S tem dodatkom smo olajsali dostop do velike zbirke podatkov Svetovne banke,
% ki jih lahko sedaj znotraj programa Orange za svoje delo uporabi kdorkoli.
%
%- lazje vzdrzevanje in posodabljanje kode
%
%
%
%Tako smo omogocili lazje


Nas graficni dodatek za dostop do podatkov indikatorjev lahko nadgradimo tako,
da uporabnikom graficnega vmesnika omogocimo vecjo izbiro oblik izhodnih
podatkov in natancnejse presejanje rezultatov. Dodamo lahko tudi vec
metapodatkov na posamezne stolpce Orange tabele, ki nam omogocijo boljso
predstavnost v ostalih Orange gradnikih. V graficni vmesnik za dostop do
podnebnih podatkov lahko dodamo se moznost izbire vodotocnih obmocji meritev.
Za boljso predstavo bi lahko postopek izbire drzav, regij in vodotocnih
obmocij omogocili prek interaktivnega zemljevida sveta.
