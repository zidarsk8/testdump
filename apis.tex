\chapter{Spletni viri indikatorjev dr"zav sveta}

Na spletu je mnogo prosto dostopnih virov oz. baz podatkov. Ti imajo programske
vmesnike, ki omogo"cajo dostop do raznovrstnih podatkov, kot so npr. seznami 
stopnje ogrozenosti zivali po drzavah 
\fnurl{http://apiv3.iucnredlist.org/api/v3/docs},
Nasini podatki meritev in slike vesolja
\fnurl{https://api.nasa.gov/#getting-started},
seznam knjig z ocenami in povezavami med uporabniki
\fnurl{https://www.goodreads.com/api},
zgodovina meteorolo"skih meritev
\fnurl{http://climatedataapi.worldbank.org/},
razni indikatorji stopnje razvoja dr"zav
\fnurl{http://api.worldbank.org/}.
Pri nalogi smo se osredoto"cili na dva programska vmesnika za dostop podatkov 
Svetovne banke, to sta zgodovina meteorolo"skih meritev (4) in razni 
indikatorji stopnje razvoja dr"zav (5).


% 1 IUCN 2016. IUCN Red List of Threatened Species. Version 2016-1 <www.iucnredlist.org>
% http://apiv3.iucnredlist.org/api/v3/docs
% 
% 2 https://api.nasa.gov/#getting-started
% 
% 3 https://www.goodreads.com/api
% 
% 4 http://climatedataapi.worldbank.org/
% 
% 5 http://api.worldbank.org/


Za podatkovno bazo Svetovno banko smo se odlo"cili, ker zdru"zuje in na enovit
na"cin predstavi podatke iz ve"c razli"cnih virov. Podatkovni viri za indikatorje
stopnje razvoja dr"zav so:

- World Development Indicators, 
- Global Development Finance, 
- African development Indicators, 
- Doing Business,
- Enterprise Surveys, 
- Millennium Development Goals, 
- Education Statistics, 
- Gender Statistics,
- Health and Nutrition Statistics, 
- IDA Results Measurement System.

Podatkovni viri za klimatske meritve so pridobljeni s svetovnih meteorolo"skih 
postaj.


Dostop do podatkov je omogo"cen prek vmesnika REST, ki ponuja veliko mo"znosti 
za iskanje in presejanje rezultatov.





\section{Indikatorji razvoja dr"zav}

Programski vmesnik indikatorjev razvoja dr"zav Svetovne banke omogo"ca dostop
do prek 16.000 raznih indikatorjev. Podatki indikatorjev so merjeni od leta
1960 dalje. Ta vmesnik omogo"ca filtriranje po naslednjih poljih:

- MRV (most recent value) - nekaj zadnjih meritev
- frequency - pogostost vzor"cenja (letno, "cetrtletno, mese"cno)
- gapfill - manjkajo"ce vrednosti prej"snjih meritev
- date - datum ali obdobje
- page - stran
- per\_page - "stevilo elementov na stran





% programski vsebuje filtre za
% MRV = most recent value
% frequency 
% gapfill
% date
% page
% per\_page



\subsection{Opis programskega vmesnika indikatorjev}


Programski vmesnik za podatke o indikatorjih razvoja omogo"ca 


\subsection{Dostop do podatkov indikatorjev}

Dostop do podatkov je omogo"cen preko dostopne to"cke

http://api.worldbank.org

Tukaj najdemo vmesnike za iskanje indikatorjev (indicators), dr"zav in regij
(countries), raven dohodka (Income level), vrsta posojila (lending type), tem
(topic), virov (sources).





\section{Te"zave pri dostopu}

- posodobitve spletne strani ter podatkovnega vmesnika, 404 ve"cina strani.

- Pomanjkljiva dokumentacija
    - polje za datum je opisano vendar ni dokumentirano kak"sne so vse mo"zne
      vrednosti. 

- manjkajo"ci identifikatorji za polja na naklju"cnih indikatorjih.
  primer (dr"zava ima le ime in prazno id polje)
- datum vsebuje naklju"cne vrednosti (last known value 2001 - 2015, 2040)
- nedokumentira meja "stevila izbranih lokacij 250
- slabo definirano delovanje filtrov podatkov (mrv in date in fill)
- Nemogo"ce ugotoviti ali ima indikator letne, "cetrtlente ali mese"cne vrednosti.



CCCCC
