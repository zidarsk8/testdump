%%%%%%%%%%%%%%%%%%%%%%%%%%%%%%%%%%%%%%%%
% povzetek 
\addcontentsline{toc}{chapter}{Povzetek}
\chapter*{Povzetek}


\textbf{Naslov:} Dostop do podatkov Svetovne banke v orodju Orange

\ \\
\textbf{Avtor:} Miha Zidar

\ \\
Program Orange je prosto dostopno orodje za podatkovno rudarjenje, s katerim
lahko za namene analiz uporabimo različne podatkovne vire. Sam program Orange
vsebuje lastne testne predpripravljene podatkovne vire, podobne vire si lahko 
pripravi in uvozi tudi uporabnik sam, ali pa uporabi katerega od ze obstojecih
dodatkov za uvoz podatkov. Za namen naloge smo izdelali dodatek Orange data sets (ODS),
s katerim je mogoce prebrati podatke s prosto dostopnega programskega vmesnika 
(API) Svetovne banke (SB). Trenutno Svetovna banka omogoca uporabo stirih razlicnih API-jev
(gospodarski indikatorji (time-series), projekti SB, financni in klimatski podatki). Dodatek ODS
je namenjen lazjemu branju in pretvorbi podatkov indikatorjev in klimatskih podatkov.
S tem bo uporabnikom programa Orange omogocena enostavnejsa uporaba velikega stevila
podatkov iz omenjenih dveh programskih vmesnikov.

\ \\
\textbf{Ključne besede:} Podatkovno rudarjenje, programski vmesnik, 
svetovna banka, gospodarski indikatorji, podnebni podatki, Orange. 




% prazna stran
\clearemptydoublepage

%%%%%%%%%%%%%%%%%%%%%%%%%%%%%%%%%%%%%%%%
% abstract
\selectlanguage{english}
\addcontentsline{toc}{chapter}{Abstract}
\chapter*{Abstract}



\subsubsection*{Keywords:}

rainbow

\selectlanguage{slovene}
