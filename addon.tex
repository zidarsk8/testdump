
\chapter{Knjižnica in gradniki za Orange}

V okviru diplomske naloge smo razvili tri ločene komponente za programerje in
končne uporabnike programa Orange. Prva komponenta je
dostopna kot samostojni paket 
\verb|simple_wbd|\fnurl{https://pypi.python.org/pypi/simple_wbd/0.5.1}. 
Druga in tretja komponenti pa sta združeni v paketu
\verb|orange3-datasets|\fnurl{https://pypi.python.org/pypi/Orange3-Datasets/0.1.3}.

Prva komponenta je programska knjižnica \verb|simple_wbd|, ki
omogoča enostaven dostop do programskega vmesnika indikatorjev in podnebnih
podatkov Svetovne banke. Knjižnico smo implementirali z uporabo čim manjšega
števila odvisnosti in je 
% TODO ali je python z veliko
namenjena splošni uporabi v programih napisanih v jeziku Python. Poudarka pri zasnovi knjižnice 
\verb|simple_wbd| sta predvsem enostavnost razširitve in zanesljivost. Ta cilja
dosežemo z mehanizmom za vključevanje lastne kode v komponente knjižnice
in mehanizmi za popravljanje ali odstranjevanje pokvarjenih podatkov.

Drugi sestavni del je razširitev knjižnice \verb|simple_wbd| s 
funkcionalnostmi, potrebnimi za lažje delo v programu Orange. To predvsem 
zavzema pretvorbo pridobljenih podatkov v podatkovno tabelo Orange in tabelo 
numpy. Ta sklop je namenjen skriptnemu delu s programom Orange 
\cite{orange_scripting} in je dostopen
kot \verb|api_wrapper| Python modul znotraj paketa \verb|orangecontrib.wbd|.

Tretji sestavni del je grafični vmesnik za uporabo \verb|api_wrapper| modula.
Namen grafičnega vmesnika je omogočiti laikom dostop do podatkov 
programskega vmesnika Svetovne banke znotraj programa Orange za namen obdelave,
analize in iskanja zakonitosti v podatkih.

\section{Knjižnica simple\_wbd}

Knjižnica \verb|simple_wbd| programerjem olajša dostop do podatkov 
programskega vmesnika Svetovne banke. Glavni namen knjižnice je 
združevanje večjega števila zahtev po podatkih in enostavna predstavitev 
prejetih rezultatov. Te rezultate je nato iz več dimenzij možno pretvoriti v 
dvo-dimenzionalno polje, primerno za uporabo v programu Orange. Glavna 
razreda te knjižnice sta \verb|IndicatorAPI| in \verb|ClimateAPI|. Prvi 
omogoča pridobivanje podatkov iz programskega vmesnika indikatorjev, drugi pa 
s programskega vmesnika podnebnih meritev.


Čeprav za dostop do programskega vmesnika Svetovne banke že obstajajo 
rešitve kot sta knjižnici 
\verb|wbdata|\fnurl{https://pypi.python.org/pypi/wbdata/0.2.7} in
\verb|wbpy|\fnurl{https://pypi.python.org/pypi/wbpy/2.0.1}, smo se odločili
za lastno implementacijo podobne knjižnice. Glavni razlog za to je, da
obstoječe rešitve poskušajo čim bolj natančno predstaviti programski
vmesnik Svetovne banke, ne pa olajšati dostop do čim večje količine
podatkov.

Za potrebe te knjižnice smo razvili lastno rešitev za predpomnjenje poizvedb,
saj so se bolj splošne rešitve, kot na primer
\verb|vcrpy|\fnurl{https://pypi.python.org/pypi/vcrpy/1.10.0} in
\verb|requests-cache|\fnurl{https://pypi.python.org/pypi/requests-cache/0.4.12}, 
izkazale za prepočasne ko delamo z večjimi količinami podatkov. Naša
rešitev za predpomnjenje izkorišča dejstvo da je vsaka poizvedba določena le
z naslovom URL, in da so vsi odgovori oblike JSON. Za vsak URL naredimo novo
datoteko v sistemskem začasnem imeniku, v kateri hranimo serializirane JSON
podatke. Ker se podatki na programskem vmesniku Svetovne banke redko
posodabljajo, smo za čas veljavnosti začasnih datotek izbrali en teden.

% NOTE: serailizacija je baje uredu beseda:
% http://eprints.fri.uni-lj.si/2711/1/63100205-VALENTIN_KRAGELJ-\
% Pregled_in_analiza_tehnologij_za_serializacijo_objektov.pdf


\subsection{Razred IndicatorAPI}
\label{razered_indicator_api}

\verb|IndicatorAPI| je razred namenjen pridobivanju podatkov indikatorjev
razvoja držav. Ker ima programski vmesnik Svetovne banke omejitev koliko 
podatkov lahko prenesemo z eno poizvedbo in nam dovoli tvoriti poizvedbe le za
en indikator na enkrat, smo napisali razred, ki v ozadju tvori in izvede
poizvedbe za vse strani vseh zahtevanih indikatorjev. To poskrbi tako da se po
prvi poizvedbi za en indikator sprehodi čez število preostalih strani 
(Primer \ref{basic_response}), ki so na voljo, in pridobljene podatke več
strani združi in predstavi kot rezultat ene same poizvedbe. Ta postopek ponovi
za vse zahtevane indikatorje, in njihove rezultate vrne v obliki slovarja, ki 
ima za ključ kodo indikatorja posamezne zahteve.

Poleg tega da skrbi za prenos vseh strani podatkov, tudi beleži število 
izvedenih in število potrebnih poizvedb za celoten prenos. Ta števila se
lahko uporablja za prikaz napredka prenosa podatkov.


Za namene razreda \verb|IndicatorAPI| smo v knjižnici \verb|simple_wbd|
razvili mehanizme za odpravo nekaterih napak omenjenih v poglavju 
\ref{api_gotchas}.

Pri manjkajočih vrednostih držav v poizvedbah za podatke indikatorjev,
poskušamo določiti pravilne vrednosti. To naredimo s pomočjo dveh
slovarjev: prvi slika kode držav v imena, drugi pa imena držav v kode. V
primeru manjkajoče vrednosti kode ali imena, poskušamo to prebrati iz enega
od naštetih slovarjev. Če nam ne uspe ugotoviti manjkajočih vrednosti,
trenutni vnos odstranimo iz rezultata poizvedbe.

Drugi tip napak, ki ga lahko delno popravimo, so napačne vrednosti v polju
\verb|date| v poizvedbah za podatke indikatorjev. Ker lahko v temu polju
pričakujemo poljubno besedilo, dela naš pretvornik za polje \verb|date| v 
datum, tako da poskuša v datum pretvoriti čim daljšo predpono besedila.
Če nam ne uspe besedila pretvoriti v veljaven datum, trenutni vnos odstranimo
iz rezultata poizvedbe.


\ \\
Glavne metode ki jih ponuja razred IndicatorAPI so:

\begin{description}  
\item [get\_indicators] za pridobivanje seznama indikatorjev s kodami, imeni
      in opisi,
\item [get\_countries] za pridobivanje seznama držav z metapodatki,
\item [get\_dataset] za pridobivanje instance razreda \verb|IndicatorDataset|,
      ki vsebuje podatkov indikatorjev.
\end{description}

Ena izmed lastnosti razreda \verb|IndicatorAPI| je, da mu lahko ob
inicializaciji podamo razred v katerem želimo prejeti rezultat poizvedbe. Ta
razred mora dedovati od osnovnega razreda \verb|IndicatorDataset|. Na ta
način lahko enostavno razširimo funkcionalnost \verb|simple_wbd| knjižnice.
V primeru \ref{indicator_api_extend} vidimo en način za razširitev razreda 
\verb|IndicatorDataset| tako da uporabniku razreda \verb|MyIndicatorAPI| ni
potrebno izrecno podati razreda \verb|IndicatorDataset| v konstruktor.

\begin{snippet}
\begin{center}
\begin{lstlisting}
class MyIndicatorDataset(simple_wbd.IndicatorDataset):
    
    def as_numpy(self):
        raise NotImplemented()
    
    def as_orange_table(self):
        raise NotImplemented()

class MyIndicatorAPI(simple_wbd.IndicatorAPI):

    def __init__(self):
        super().__init__(MyIndicatorDataset)
\end{lstlisting}
\end{center}
\caption[some]{Primer razširitve osnovnega razreda rezultatov poizvedb.}
\label{indicator_api_extend}
\end{snippet} 


\subsubsection{Razred IndicatorDataset}
\label{razred_indicatordatasets}

Razred \verb|IndicatorDataset| je osnovni razred v katerem dobimo zahtevane 
podatke indikatorjev. Ta razred vsebuje vse potrebne metode in podatke za 
predstavitev rezultatov programskega vmesnika, na dva načina: kot slovar
rezultatov poizvedb za posamezen indikator in dvo dimenzionalen seznam. 
Posamezna vrednost v teh podatkih je določena z državo, časovno komponento in 
kodo indikatorja. 

Podatke lahko predstavimo kot dvodimenzionalno polje v dveh oblikah: kot
časovne vrste ali kot podatki držav. Obliko predstavitve izberemo s
parametrom \verb|time_series| metode \verb|as_list|. Za predstavitev obeh oblik
je prva vrstica polja uporabljena kot naslovna vrstica, ki opisuje podatke v 
stolpcih.

Ko uporabljavo obliko časovnih vrst, so elementi prve vrstice kartezični
produkt kod indikatorjev in držav. V prvem stolpcu polja pa imamo časovno
komponento podatkov. Na ta način so vsi ostali elementi polja določeni s 
časovno komponento, državo in kodo indikatorja.

Ko dostopamo do dvodimezionalnega polja ki predstavlja podatke držav, pa je v
prvi vrstici kartezični produkt kod indikatorjev in časovne komponente. Prvi
stolpec v tej predstavitvi vsebuje imena držav. Za razliko od predstavitve v 
obliki časovnih vrst, v to polje vstavimo se dodatne stolpce ki vsebujejo
metapodatke drzav iz primera \ref{country_response}: regija \verb|region|, 
administrativna regija \verb|adminregion|, višina dohodka \verb|incomeLevel|, 
vrsta posojil \verb|lendingType|, geografska širina \verb|latitude|, 
geografska dolžina \verb|longitude|. Tudi tukaj vsi ostali elementi določeni s
časovno komponento, državo in kodo indikatorja. 


\subsection{Razred ClimateAPI}

Razred \verb|ClimateAPI| olajša dostop do podnebnih podatkov programskega
vmesnika Svetovne banke. Ta programski vmesnik dovoli poizvedbe po podatkih le 
ene vrste meritev za eno vrsto meritvenega obdobja in eno državo. Naš razred 
naredi kartezični produkt med vsemi zahtevanimi vrstami meritev, vrstami
meritvenih obdobij in državami. Nato iz tega zgradi in izvede vse poizvedbe 
in predstavi podatke kot enotni odgovor. V razredu \verb|ClimateAPI| hranimo 
tudi število vseh potrebnih poizvedb in število že izvedenih poizvedb, kar 
lahko uporabimo za prikaz napredka prenosa podatkov.



\subsubsection{Razred ClimateDataset}

Razred \verb|ClimateDataset| je osnovni razred v katerem dobimo zahtevane 
podatke podnebnih meritev. Ta razred vsebuje vse potrebne metode in podatke za 
predstavitev rezultatov programskega vmesnika, na dva glavna načina: kot
gnezden slovar in dvo dimenzionalen seznam. Posamezna vrednost v teh podatkih
je določena z državo, vrsta podatkov, in časovno komponento. Poleg omenjenih
načinov predstavitve podatkov lahko dostopamo tudi do neobdelanih podatkov 
prejetih iz programskega vmesnika za vsako poizvedbo posebej.

Časovno komponento rezultata sestavljata vrsta meritvenega obdobja in 
začetek obdobja meritve. Sestavljeno časovno komponento uporabljamo, da se 
izognemo dvoumnim primerom vrednosti začetka obdobja za letni in desetletni 
interval meritev. Primera takih dveh časovnih obdobij sta 
\verb|'decade - 1990'| in \verb|'year - 1990'|.


Do podatkov predstavljenih z gnezdenim slovarjem lahko dostopamo preko funkcije
\verb|as_dict|. V tej funkciji združimo podatke poizvedb programskega
vmesnika v gnezden slovar s štirimi nivoji gnezdenja: država, vrsta meritev,
vrsta meritvenega obdobja in obdobje meritve. Zadnji nivo gnezdenja vsebuje
vrednosti podnebnih meritev.

Pri predstavitvi podatkov kot dvodimenzionalno polje, moramo dve od treh
komponent podatkov (država \verb|'country'|, vrsta podatkov \verb|'type'|, 
in časovna komponenta \verb|'interval'|)
združiti in ju skupaj prikazati v vrsticah ali stolpcih. Za razliko od razreda
\verb|IndicatorDataset|, ki podpira le dve obliki prikaza, lahko v razredu
\verb|ClimateDataset| sami določimo katere komponente bodo v stolpcih in
katere v vrsticah. Primer različnih izborov komponent je prikazan v
\ref{list_configurations}. Spremenljivki \verb|list1| ind \verb|list2| iz
prejšnjega primera prikazujeta privzeto konfiguracijo, kjer imamo v stolpcih
kartezični produkt vrst meritev in vrst meritvenih obdobij, v vrsticah pa
podatke države. Spremenljivka \verb|list4| prikazuje konfiguracija za
predstavitev v obliki časovnih vrst.

\begin{snippet}
\begin{center}
\begin{lstlisting}
import simple_wbd

api = simple_wbd.ClimateAPI()                   
climate_dataset = api.get_instrumental(['svn', 'usa', 'aus'])

list1 = ds.as_list()
list2 = ds.as_list(columns=['type', 'interval'])  # default  value
list3 = ds.as_list(columns=['type'])
list4 = ds.as_list(columns=['type', 'country']) 
list5 = ds.as_list(columns=['country'])
\end{lstlisting}
\end{center}
\cprotect
\caption{Prikaz nekaj možnih oblik dvodimezionalnega polja vrednosti.} 
\label{list_configurations}
\end{snippet} 



\section{Modul api\_wrapper}


Znotraj paketa \verb|orangecontrib.wbd| smo razvili modul \verb|api_wrapper| v
katerem smo razširili razreda \verb|IndicatorDataset| in \verb|ClimateDataset|
na način ki je prikazan v primeru \ref{indicator_api_extend}. Naša
razširitev obema razredoma doda metodi za pretvorbo podatkov v podatkovno 
tabelo Orange in tabelo numpy.

\subsection{Razširitev razreda IndicatorDataset}

Glavne funkcionalnosti, za uporabo programske vmesnika indikatorjev, so
vključene v naši razširitvi razreda \verb|IndicatorDataset|. To je na prvem
mestu metoda \verb|as_numpy_array|, ki rezultat metode \verb|as_list|
opisane v poglavju \ref{missing}, spremeni v polje numpy in odstrani vse
stolpce, ki ne vsebujejo niti ene veljavne vrednosti. Druga metoda pa je
\verb|to_orange_table|, ki podatke dobljene iz metode \verb|as_numpy_array|,
pretvori v podatkovno tabelo Orange. To tabelo lahko oblikuje kot "casovno
vrsto ali pa kot seznam dr"zav, kot je opisano v poglavju 
\ref{razred_indicatordatasets}. Katero obliko tabele Orange "zelimo izbrati,
dolo"cimo s parametrom \verb|time_series|. % TODO: (angl.\ Orange data table).
Ta metoda tudi poskrbi za pravilno nastavljeno domeno\footnote{Domena
``Domain'' je razred v orodju Orange, ki določa tipe in imena značilk in
ciljnih razredov.} podatkov.


\subsection{Raz"siritev razreda ClimateDataset}
\label{razsiritev_razreda_climatedataset}

Prav tako kot raz"siritev razreda \verb|IndicatorDatasets|, tudi ta raz"siritev
doda metodi \verb|as_numpy_array| in \verb|to_orange_table|. Prav tako kot v
raz"siritvi razreda \verb|IndicatorDatasets|, lahko tudi tukaj s parametrom
\verb|time_series| izberemo obliko tabele Orange. Pri vrednosti parametra 
\verb|time_series = False| se nastavi privzeta oblika tabele, prikazana kot
\verb|list1| sicer pa kot \verb|list3|, iz primera \ref{list_configurations}.
S tem parametrom pa izgubimo mo"znost poljubne oblike orange tabele.


\section{Grafični vmesnik}

Za namen tega diplomskega dela smo z dodatkom \verb|Orange3-DataSets|
grafičnemu vmesniku programa Orange, dodali novo skupino gradnikov imenovano
``Data Sets'' (Slika \ref{data_sets_group}). V okviru te naloge smo za skupino
``Data Sets'' izdelali dva ločena gradnika. Prvi gradnik se imenuje ``WB
Climate'' (Slika \ref{co2_temp_climate}) in nam preko grafičnega vmesnika 
omogoča dostop do podnebnih podatkov Svetovne banke. Drugi gradnik pa se 
imenuje ``WB Indicators'' (Slika \ref{co2_temp_indicator}) in nam
preko grafičnega vmesnika omogoča dostop do podatkov indikatorjev razvoja.

Oba grafi"cna vmesnika sta narejen skladno z vodili grafi"cnih vmesnikov
programa Orange. To smo dosegli tako da smo za ve"cino elementov grafi"cnega
vmesinka, uporabili predpripravljene gradnike v paketu \verb|Orange.gui|. Pri
gradnji teh vmesnikov pa smo bili pozorni na odzivnost grafi"cnega vmesnika, in
smo po"casne operacije branja podatkov z interneta prestavili v lo"ceno nit.
 
\begin{figure}
  \begin{center}
    \includegraphics[width=4cm]{pic/data_sets_group.png}
  \end{center}
  \caption{Skupina gradnikov data sets}
  \label{data_sets_group}
\end{figure} 


\subsection{Gradnik WB Indicators}

\verb|WB Indicators| je gradnik programa Orange za dostop do podatkov
programskega vmesnika indikatorjev. Ta gradnik nam omogo"ca enastavno izbiro
enega ali ve"c indikatorjev in ene ali ve"c dr"zav za katere "zelimo dobiti
podatke izbranih indikatorjev. Za la"zje iskanje indikatorjev smo v grafi"cnem
vmesniku dodali dve mo"znosti presejanja seznama indikatorjev. Pri prvem situ
si lahko izberemo prikaz vseh indikatorjev, pogosto uporabljenih 
indikatorjev\footnote{Seznam je na voljo na strani 
\url{http://data.worldbank.org/indicator?tab=all}}
ali pa izpostavljenih indikatorjev\footnote{Seznam je na voljo na strani 
\url{http://data.worldbank.org/indicator?tab=featured}}. Drugo sit pa je
tekstovno presejanje po poljih: koda ``Id'', ime ``Name'', teme ``Topics'' in
viri ``Sources''. 
% TODO: preveri ta stavek
V grafi"cnem vmesniku si lahko tudi izberemo eno izmed oblik izhodnih podatkov
kot "casovne vrste \angl{Time Series} ali podatke dr"zav \angl{Countries}, 
kot smo ju opisali v razdelku \ref{razred_indicatordatasets}. 
Implementirali pa smo tudi prikaz napredka prenosa podatkov, s "stevili vseh in
"ze izvedenih poizveb, omenjeni v razdelku \ref{razered_indicator_api}.


\begin{figure}
\begin{center}
\includegraphics[width=12cm]{pic/co2_temp_indicator_selection.png}
\end{center}
\caption{Prikaz gradnika WB Indicators.}
\label{co2_temp_indicator}
\end{figure} 




\subsection{Gradnik WB Climate}

Gradnik za izbiro podnebnih podatkov podatkovnega vmesnika Svetovne banke, nam
ponuja mo"znosti izbire dr"zav, vrste podatkov in vrste meritvenega obdobja.
Prav tako kot gradnik WB Indicators, lahko tudi tukaj izberemo obliko izhodnih
podatkov. Mo"zni izbiri oblike izhodnih podatkov sta "casovne vrste in podatki
dr"zav, kot smo opisani v poglavju \ref{razsiritev_razreda_climatedataset}. 
Kot dodatno mo"znost pa imamo v tem grafi"cnem vmesniku tudi zastavico, ki 
dolo"ca ali bomo za dr"zave izpisovali imena ali pa kode. Tudi temu grafi"cnemu
vmesniku smo dodali prikaz napredka prenosa podatkov.



\begin{figure}
\begin{center}
\includegraphics[width=12cm]{pic/co2_temp_climate_selection.png}
\end{center}
\caption{Prikaz gradnika WB Climate.}
\label{co2_temp_climate}
\end{figure} 
