
\chapter{Uvod}

Na svetovnem spletu je dosegljivih vedno ve"c prosto dostopnih programskih
vmesnikov (ang.\ application programming interface). 
% TODO: morda dodaj API v razlago zgoraj
Ti vmesniki omogo"cajo dostop
do zelo raznolikih zbirk podatkov. Nekaj primerov prosto dostopnih podatkovnih
zbirk je seznam stopnje ogro"zenosti "zivali po dr"zavah 
% 1 IUCN 2016. IUCN Red List of Threatened Species. Version 2016-1 <www.iucnredlist.org>
    \fnurl{http://apiv3.iucnredlist.org/api/v3/docs},
podatki meritev in slike vesolja agencije NASA
    \fnurl{https://api.nasa.gov/},
seznam knjig z ocenami in povezavami med uporabniki 
    \fnurl{https://www.goodreads.com/api},
zgodovina meteorolo"skih meritev 
    \fnurl{http://climatedataapi.worldbank.org/},
razni indikatorji stopenj razvoja dr"zav
    \fnurl{http://api.worldbank.org/}.

Programski vmesniki so oblikovani tako, da je omogo"cena raznolika uporaba
podatkov iz podatkovnih zbirk. To pa ima tudi slabost, ki je v tem, da je 
podatke potrebno predhodno obdelati za vsak namen posebej. Tako bi na primer 
moral vsak uporabnik programa Orange podatke predhodno pretvoriti v obliko, 
primerno za njegovo konkretno analizo.


\section{Motivacija}

% TODO: tri motivacije: 
% lazja uporaba laikom - gui klikanje,
% lazje vzrdzevanje in posodobobitve, en codebase,
% ne rabi usak delat iste stvari, rocno gradit tab separate fajle, manj napak,
%   usak skrbeti za iste pomankljivosti apija (slabi podatki itd).

Povezava programskega vmesnika za dostop do podatkov in orodja za analizo 
podatkov je pogosto prezapletena za navadnega uporabnika. Z dodatkom Orange
data sets "zelimo podatke programskega vmesnika Svetovne banke spraviti v 
obliko, primerno za nadaljnjo
uporabo v orodju Orange. Ta dodatek bi pomagal zdru"ziti programe za obdelavo
podatkov in prosto dostopne zbirke podatkov. S tem dobimo enostavnej"si dostop 
do podatkov iz prek 16.000 indikatorjev in "stevilnih podnebnih meritev,
s "cimer bomo la"zje analizirali in iskali morebitne zakonitosti v podatkih.
"Ce bi imeli en sam ustrezen dodatek za dostop do podatkov programskega 
vmesnika Svetovne banke, bi poenostavili posodabljanje in
vzdr"zevanje kode v primeru sprememb programskega vmesnika za vse uporabnike
istega orodja hkrati. S tem odpravimo potrebo, da bi moral vsak uporabnik sam
skrbeti za uskladitvene posodobitve.


\section{Cilji in struktura diplomske naloge}

Cilj diplomske naloge je izdelati knji"znico za uporabo programskega vmesnika
Svetovne banke ter izdelati dodatek za program Orange, ki s pomo"cjo omenjene
knji"znice omogo"ca uporabniku dostop do podatkov Svetovne banke preko
grafi"cnega vmesnika.

V diplomski nalogi najprej predstavimo spletna vira indikatorjev
dr"zav sveta in meritev podnebnih podatkov Svetovne banke, ter
opi"semo delovanje njunih programskih vmesnikov.
Nato podrobneje opi"semo na"so implementacijo knji"znice za dostop do
programskega vmesnika Svetovne banke in gradnikov za program Orange, ki to
knji"znico uporabljajo. V nadaljevanju prika"zemo "se nekaj prakti"cnih 
primerov uporabe dodatka Orange data sets. Na koncu "se popi"semo opravljeno 
delo, navedemo vire kode in omenimo mo"zne na"cine za izbolj"savo ali 
nadgradnjo na"sega dodatka.












% bomo podrobneje opisali programski vmesnik za dostop do
% podatkov Svetovne banke (API SB). V "cetrtem poglavju sledi predstavitev
% knji"znice in gradnikov za Orange in nato "se konkretni primeri uporabe. Na koncu
% bomo popisali opravljeno delo, navedli vire kode in opisali nadaljne mo"znosti
% nadgradnje dodatka.


%% -------------------------------
%%
%%  - Prav tako se ve"cina programov in knji"znic za dostop do baz podatkov osredoto"ci le na iskanje po teh bazah, ne pa tudi na pridobivanje "cim ve"cje koli"cine podatkov.
%%
%%
%% % Poleg tega obstaja dosti odprto kodnih programov za obdelavo
%% % in analizo podatkov. Ker so programski vmesnike bolj splo"sno namenski, je
%%
%%
%% \chapter{Uvod}
%%
%%
%% Na spletu je vedno ve"c prosto dostopnih programskih vmesnikov(API, ang.
%% application programming interface) za razli"cne baze podatkov. Ti vmesniki
%% ponujajo dostop do zelo raznolikih podatkov, kot so
%% seznami stopnje ogro"zenosti "zivali po dr"zavah
%% \fnurl{http://apiv3.iucnredlist.org/api/v3/docs},
%% NASA podatki meritev in slike vesolja
%% \fnurl{https://api.nasa.gov/},
%% seznam knjig z ocenami in povezavami med uporabniki
%% \fnurl{https://www.goodreads.com/api},
%% zgodovina meteorolo"skih meritev
%% \fnurl{http://climatedataapi.worldbank.org/},
%% razni indikatorji stopnje razvoja dr"zav
%% \fnurl{http://api.worldbank.org/}.
%%
%% Ker pa so programski vmesniki bolj splo"sno namenski, je
%% podatke te"sko spraviti v obliko ki bi bila primerna za uporabo v raznih
%% orodjih za analizo in obdelavo podatkov. Prav tako se ve"cina programov in knji"znic za dostop
%% do baz podatkov osredoto"ci le na iskanje po teh bazah, ne pa tudi na
%% pridobivanje "cim ve"cje koli"cine podatkov.
%%
%%
%% % Poleg tega obstaja dosti odprto kodnih programov za obdelavo
%% % in analizo podatkov. Ker so programski vmesnike bolj splo"sno namenski, je
%% % podatke te"sko spraviti v obliko za analizo in obdelavo. Z orodjem, ki bi
%% % pomagalo zdru"ziti programe za obdelavo podatkov in prosto dostopne baze
%% % podatkov, bi omogo"cili raziskovanje teh podatkov "sir"si javnosti.
%%
%%
%%
%% % Povezava programskega vmesnika in orodja za analizo podatkov pa je pogosto
%% % prezapletena za navadnega uporabnika. Z dodajanjem gradnikov za enostavno
%% % uporabo spletnih programskih vmesnikov v orodjih kot je Orange, omogo"cimo ...
%%
%% \section{Motivacija}
%%
%% Branje podatkov z raznih programskih vmesnikov je lahko zelo zamudno delo.
%% Programski vmesnik se lahko s "casom spremeni, in podatki ki jih dobimo z
%% vmesnika so lahko pokvarjeni. Trenutni pristop, kjer moramo podatke vsaki"c ro"cno
%% obilkovati da so primerni za analizo, ima mnogo pomanjkljivosti. Prejeti podatki
%% lahko vsebujejo nepravilnosti, ali pa so celo nedostopni. Z dodatkom ki bi
%% poskrbel za prenos podatkov in pretovrbo v uporabno obliko, hkrati pa bi znal
%% popraviti ali odstraniti pokvarjene podatke, bi lahko ve"c pozornosti posvetili
%% sami analizi in obdelavi. Poleg tega, pa ve"cina knji"znic za delo z odprtimi
%% programskimi vmesniki, nudi zelo dobre na"cine za iskanje posameznih podatkov,
%% ne pa za prenos ve"cje koli"cine podatkov kar je bolj primerno za analizo in
%% obdelavo.
%%
%%
%%
%% \section{Cilji}
%%
%% Cilj diplomske naloge je izdelati knji"znico, ki omogo"ca enostaven dostop do
%% podatkov Svetovne banke in interaktivni gradnik v programu Orange za dostop in
%% uporabo teh podatkov. S tem bomo omogo"cili raziskovanje teh podtakov "sir"si
%% javnosti. Knji"znica bo poenostavila prenos ve"cjega "stevila podatkov, in
%% predstavila te podatke v obliki primerni za orodje Orange. k<
